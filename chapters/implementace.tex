\chapter{Implementace}

\todo{Tohle je ještě hodně v počátečním stavu, stačí se kouknout velký chyby}

Tato kapitola se věnuje samotné implementaci všech komponent, které umožní funkčnost testovací knihovny.

\section{Komunikace}
Testovací knihovna má jasně definovanou komunikaci. Strukturu jedné zprávy zajišťuje třída \inlinecode{Message}. \todo{Popis třídy Message}

\section{Nastavení knihovny}
Ke správnému fungovaní knihovny je zapotřebí určitá konfigurace. Tato konfigurace se dá rozdělit na tři části:
\begin{enumerate}
    \item Konfigurace služby
    \item Konfigurace frameworku MSTest s testovací knihovnou
    \item Konfigurace propojení se serverem Azure DevOps
\end{enumerate}

V této části se budeme věnovat pouze konfiguraci služby. Ostatní konfigurace budou probrány v dalších částích. Konfigurace je uložena v souboru \inlinecode{config.xml}. Ukázku tohoto souboru můžeme vidět na obrázku \ref{listing:configxml}. Tento soubor obsahuje tři hodnoty pro nastavení:

\begin{description}
    \item[IP] Adresa, na které bude služba poslouchat příchozí připojení. Všechna zařízení se budou připojovat na tuto adresu.
    \item[Port] Síťový port, skrz který služba provádí komunikaci
    \item[Participants] Počet fyzických participantů, které se připojí ke službě.   
\end{description}

\begin{listing}
    \begin{minted}{xml}
        <?xml version="1.0" encoding="UTF-8" ?>
        <configuration>
          <!-- IP of the service-->
          <ip>192.168.4.100</ip>
          <!-- Port of the service -->
          <port>1337</port>
          <!-- Number of non-virtualized participants-->
          <participants>1</participants>
        </configuration>
    \end{minted}
    \caption{Ukázka konfiguračního souboru}
    \label{listing:configxml}
\end{listing}


\section{Služba}
Jak jsem již zmínil, testování je řízeno testovací služba. V naší implementaci je tento celek rozdělen na dvě třídy. Třída \inlinecode{csharp}{TestService} obstarává jednotlivé úkony služby. Třída po konstrukci inicializuje vnitřní proměnné, ale neprovádí žádné úkony. Metodou \inlinecode{bool Init(int InitTimeout)} třída inicializuje běh služby. Proměnná \inlinecode{InitTimeout} určuje dobu, kdy služba čeká na příchozí komunikaci. 

Metoda z konfigurace zjistí IP adresu a port, na kterém má služba běžet. Následně začne na této IP adrese a portu poslouchat příchozí komunikaci. V této fázi se připojují pouze fyzická zařízení. Z nastavení služba ví, kolik participantů má očekávat. 

O přidávání participantů se stará metoda \inlinecode{int AddClient(int timeout)}. Při příchozí komunikaci je zavolána tato metoda, která přijme zařízení a vytvoří s ním připojení. Zároveň přijme identifikační zprávu od participanta. Tato zpráva má typ zprávy OK a v datech zprávy obsahuje MAC adresu participanta. Metoda odešle participantovi potvrzovací zprávu s typem zprávy OK, bez dat. Po úspěšném připojení metoda uloží vytvořeného klienta a jeho MAC adresu do seznamu připojených participantů. Metoda zároveň rozezná připojení virtuálního participanta. Tento participant má MAC adresu DE:AD:BE:EF:00:00. Nakonec metoda vrací tři hodnoty
\begin{description}
    \item[0] pokud připojování participanta vyústilo v neúspěch, ať už kvůli překročení časového limitu, nebo kvůli nedodržení stanovené komunikace.
    \item[1] pokud služba úspěšně přijme fyzického participanta
    \item[2] pokud služba úspěšně přijme virtuálního participanta
\end{description}

Služba úspěšně ukončuje inicializační fázi poté co se úspěšně připojí stejný počet participantů, jako bylo určeno v konfiguraci. V případě nepřipojení se očekávaného počtu zařízení v definovaném čase, obdržení špatné nebo žádné zprávy služba vyhodnocuje inicializační fázi jako neúspěšnou. Připojení virtualizovaného participanta v této fázi vyústí též v neúspěch inicializace.

Další metodou této třídy je metoda \inlinecode{bool AddVirtualParticipant(int timeout)}. Tato metoda může přijmout virtuálního participanta v průběhu běhu. Metoda volá stejně jako inicializační metoda metodu \inlinecode{AddClient}, které zajišťuje a ověřuje úspěšné připojení. Metoda vrací úspěch při úspěšném připojení virtuálního participanta. V případě nějaké chyby nebo připojené fyzického participanta, metoda vrací neúspěch. 

Metoda \inlinecode{TestResult RunTest<T>(T testEnum, int timeout)} zajištuje spuštění jednotlivých testů. 

