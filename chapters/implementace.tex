\chapter{Implementace}\label{chap:implementation}

V této kapitole se věnuji implementaci všech navržených komponent, které umožňují automatizaci testování.

\section{Zpráva}
Strukturu jedné zprávy zajišťuje třída \inlinecode{Message}. Diagram této třídy můžeme vidět na obrázku \ref{fig:message_class}. Třída \inlinecode{Message} je implementována v jazyce \csharp{} a \cpp{}.

Dle návrhu instance třídy obsahuje typ zprávy a data zprávy, pokud zpráva nějaké data obsahuje. Typ zprávy je určen enumerátorem {\nobreak\inlinecode{MessageType}}. Tento výčtový typ obsahuje jednotlivé typy zpráv, definované v sekci \ref{sec:communication}. Zároveň tyto typy rozšiřuje o hodnotu \inlinecode{MSG\_EXCEPTION}, která značí neplatnou zprávu.

Při odesílání a přijímání zpráv skrze protokol TCP/IP je podstatné, aby se instance třídy dala převádět na bajtové pole a obráceně. Třída \inlinecode{Message} proto obsahuje dvě metody, které tyto převody zajišťují. Jsou to:

\begin{itemize}
    \item \inlinecode{GetByteStream} -- Metoda převádí instanci třídy \inlinecode{Message} na bajtové pole.
    \item \inlinecode{GetMessageFromStream} -- Statická metoda, která v argumentu obdrží bajtové pole a vrátí instanci třídy \inlinecode{Message}.
\end{itemize}

Následná komunikace skrze protokol TCP/IP je implementována v závislosti na zařízení za pomocí těchto metod. Jednotlivé implementace třídy \inlinecode{Message} mají mezi sebou minimální rozdíly, které jsou primárně kvůli odlišnostem jazyků \csharp{} a \cpp{}.

\begin{figure}[p!]
    \centering 
    \includegraphics[width=\textwidth]{assets/img/class_diagram/message.pdf}
    \caption{Diagram třídy Message}
    \label{fig:message_class}
\end{figure}

\clearpage

\section{Testovací služba}
Testovací služba je v implementaci rozdělena na dvě hlavní třídy. Třída \inlinecode{TestService} obstarává jednotlivé úkony služby, následně třída \inlinecode{ServiceRunner} propojuje třídu \inlinecode{TestService} s ostatními komponentami, které jsou potřebné pro testování. Tyto třídy můžeme vidět znázorněné na obrázku \ref{fig:test_service}.

\begin{figure}[H]
    \centering 
    \includegraphics[width=0.75\textwidth]{assets/img/class_diagram/service.pdf}
    \caption{Diagram tříd zajištující testovací službu}
    \label{fig:test_service}
\end{figure}

\subsection{Nastavení služby}\label{sec:settings}

Konfigurace služby je uložena v souboru \inlinecode{config.xml}. Služba očekává tento soubor v kořenové složce, ze které je spouštěna. Tento soubor obsahuje tři hodnoty pro nastavení:

\begin{itemize}
    \item \inlinecode{ip} -- Adresa, na které bude služba poslouchat příchozí připojení. Všechna zařízení se budou připojovat na tuto adresu.
    \item \inlinecode{port} -- Síťový port, skrz který služba provádí komunikaci.
    \item \inlinecode{participants} -- Počet testovaných zařízení, která se připojí ke službě.   
\end{itemize}

Ukázku tohoto souboru můžeme vidět na výpisu \ref{listing:configxml}. V knihovně se o konfiguraci stará třída \inlinecode{Configuration}. Tato třída existuje v jmenném prostoru \inlinecode{GlobalVar}. Tento jmenný prostor simuluje globální proměnnou. Z tohoto důvodu jsou všechny třídy, které jsou odsud používány, konstruovány tak, aby primárně fungovaly jen ke čtení. Zároveň třída \inlinecode{GlobalVar} povolí pouze jedno přiřazení instance. Při pokusu o přiřazení jiné instance třída vyhazuje výjimku. Diagram těchto tříd můžeme vidět na obrázku \ref{fig:utility}. Třída rovněž obsahuje vlastnost \inlinecode{isConfigLoaded}, která obsahuje informaci o tom, zdali je konfigurace inicializována.

Při vytvoření instance získá třída konfiguraci z konfiguračního souboru a uloží ji do vnitřních proměnných třídy. Tyto proměnné jsou určeny pouze ke čtení, nelze je upravovat. Stav konfigurace určuje proměnná \inlinecode{bool IsValid}. V případě chyby je tato proměnná nastavena na hodnotu \inlinecode{false} a v proměnné \inlinecode{Exception} je uložen důvod neúspěchu. V opačném případě je tato proměnná nastavena na hodnotu \inlinecode{true}.

\begin{listing}[htbp]
    \centering
    \begin{cminted}{xml}
<?xml version="1.0" encoding="UTF-8" ?>
<configuration>
  <!-- IP of the service-->
  <ip>192.168.4.100</ip>
  <!-- Port of the service -->
  <port>1337</port>
  <!-- Number of non-virtualized participants-->
  <participants>1</participants>
</configuration>
    \end{cminted}
    \caption{Ukázka konfiguračního souboru}
    \label{listing:configxml}
\end{listing}


\subsection{Operace služby}

Jak jsem již zmínil, jednotlivé úkony služby jsou implementovány ve třídě \inlinecode{TestService}. 
Třída po své konstrukci inicializuje vnitřní proměnné, ale neprovádí žádné úkony. V následujících sekcích přiblížím jednotlivé dostupné operace.

\subsubsection{Odesílání a přijímání zpráv}
Třída \inlinecode{TestService} vytváří během svého běhu propojení s účastníky testování. Po úspěšném připojení služba ke komunikaci používá vestavěného TCP klienta. K této komunikaci má vytvořené dvě statické metody. Jedná se o:

\begin{itemize}
    \item \inlinecode{static void SendMessage(TcpClient client, Message msg)} \\
    Metoda pro odeslání jedné zprávy jednomu klientovi.
    \item \inlinecode{static bool RcvMessage(TcpClient client, out Message msg)} \\
    Metoda pro přijetí zprávy od jednoho klienta.
\end{itemize}

Všechny zprávy jsou následně odesílány a přijímány za pomoci těchto dvou metod. Rozšířením těchto metod jsou metody:

 {
    \setlength{\emergencystretch}{3em} 
    \begin{itemize}
        \item \inlinecode{(LibState, List<PhysicalAddress>) CheckParticipantsResponse   (MessageType expectedResponse, int timeout)} \\
        Metoda přijme od všech účastníků testování jednu zprávu a zkontroluje ji s očekávanou zprávou předanou v argumentu. Zároveň kontroluje, zdali obdrží zprávu v maximálním čase, který je definovaný argumentem \inlinecode{timeout}. 
        \item \inlinecode{(LibState, List<PhysicalAddress>) SendParticipantsMessage (Message msg)} \\
        Metoda odešle všem participantům jednu zprávu.
    \end{itemize}
 }

Obě zmíněné metody následně v návratové hodnotě vrací dvě položky. První položkou je výsledek operace. Tento výsledek je reprezentován enumerátorem \inlinecode{LibState}, jenž má tyto výčty:

\begin{itemize}
    \item \inlinecode{STATE\_OK} - označuje úspěch,
    \item \inlinecode{STATE\_FAIL} - označuje neúspěch,
    \item \inlinecode{STATE\_ERROR} - označuje fatální neúspěch, který vznikl z důvodu nějaké chyby.
\end{itemize}

Tento enumerátor je zároveň hojně využíván skrze knihovnu k reprezentaci stavu různých komponent. Druhou položkou, kterou metoda vrací, je seznam zařízení, která při komunikaci způsobila neúspěch. Tento neúspěch může vzniknout ať už obdržením jiné zprávy než očekávané v případě metody \inlinecode{CheckParticipantsResponse}, a nebo vznikem nějaké chyby při komunikaci.

\subsubsection{Inicializace}

Třída \inlinecode{TestService} pro účely inicializace a přidání klienta využívá tyto dvě metody:

\begin{itemize}
    \item \inlinecode{bool Init(int InitTimeout)} \\
    Metoda, která provádí inicializační fázi testovací služby.
    \item \inlinecode{int AddClient(int timeout)}\\
    Metoda pro přidání klienta do testovací služby.
\end{itemize}

Metodou \inlinecode{Init} třída inicializuje běh služby. Metoda z konfigurace zjistí IP adresu a port, na kterém má služba běžet. Následně začne na této IP adrese a portu poslouchat příchozí komunikaci. Z nastavení služba ví, kolik připojení má očekávat. 

Proměnná \inlinecode{InitTimeout} určuje dobu, kdy služba čeká na příchozí komunikaci. Doba je metodě předávána, stejně jako všem ostatním metodám, které mají definovaný časový limit, v milisekundách. Metoda synchronně kontroluje, zdali nějaký účastník nečeká na připojení a zdali nevypršela maximální doba na připojení. V případě příchozí komunikace metoda zavolá metodu \inlinecode{AddClient}. Pokud se chce v jeden moment připojit více účastníků, jsou ostatní účastníci zařazeni do fronty.

Po zavolání metody \inlinecode{AddClient} metoda vytvoří připojení s testovacím zařízením a následně čeká na identifikační zprávu od testovacího zařízení, nejdéle však po dobu definovanou argumentem \inlinecode{timeout}. 

Po obdržení zprávy metoda uloží vytvořeného klienta a jeho MAC adresu do seznamu připojených účastníků testování. Nakonec metoda vrací tři hodnoty typu integer:
\begin{itemize}
    \item \inlinecode{0} -- pokud připojení účastníka testování vyústilo v neúspěch, ať už kvůli překročení časového limitu, nebo kvůli nedodržení stanovené komunikace,
    \item \inlinecode{1} -- pokud se ke službě úspěšně připojí testované zařízení,
    \item \inlinecode{2} -- pokud se ke službě úspěšně připojí testovací partner.
\end{itemize}

Metoda \inlinecode{AddClient} považuje za testovacího partnera takového účastníka, který odešle jako svoji MAC adresu adresu testovacího partnera. Zařízení, která odešlou jinou MAC adresu, jsou považována za testovaná zařízení. Metoda považuje nulovou MAC adresu jako neplatnou.

Služba úspěšně ukončuje inicializační fázi poté, co se úspěšně připojí stejný počet účastníků testování, jak bylo určeno v konfiguraci. V případě nepřipojení se očekávaného počtu zařízení v definovaném čase, obdržení špatné nebo žádné zprávy služba vyhodnocuje inicializační fázi jako neúspěšnou. Připojení testovacího partnera v této fázi vyústí též v neúspěch inicializace. 

\subsubsection{Připojení testovacího partnera}

Jelikož každý test může obsahovat různý počet testovacích partnerů, je podstatné, aby služba mohla při testovacím běhu tato zařízení přidávat a odebírat. K tomu slouží dvě metody. Metoda \inlinecode{AddTestPartner} řekne službě, že má očekávat připojení testovacího partnera. V případě úspěšného připojení metoda vrací úspěch. Naopak v případě chyby, nebo připojení jiného zařízení, metoda vrací neúspěch. O opačnou operaci se stará metoda \inlinecode{StopTestPartners}. Metoda po svém zavolání odešle všem testovacím partnerům zprávu o ukončení testování a tato zařízení jsou odpojena a ukončena.

\subsubsection{Spuštění testu}

Nejpodstatnější operací je spuštění jednotlivých testů. O to se stará metoda \inlinecode{LibState RunTest<T>(T testEnum, int timeout)}, která očekává dva parametry:

\begin{itemize}
    \item \inlinecode{T testEnum} -- Identifikátor testu s generickým typem T, který typem musí být enumerátor.
    \item \inlinecode{int timeout} -- Maximální délka, po kterou služba očekává odpověď od účastníků testu. 
\end{itemize}

Metoda odesílá všem účastníkům testování direktivu k započnutí jednotlivých fází testu. Poté čeká na odpověď od všech účastníků testu. Doba čekání je určena argumentem \inlinecode{timeout}. Tento čas je vázaný na jednotlivá stádia testování. Tedy pokud má metoda časový limit 30 sekund, bude po každém odeslání direktivy k započnutí fáze testu čekat na odpověď maximálně těchto stanovených 30 sekund. Tento časový limit je realizován za pomoci metody \inlinecode{CheckParticipantsResponse}.

Metoda během běhu vyhodnocuje, zda se účastník testování nedostal do chybového stavu. Na základě toho poté spouští jednotlivé fáze. Po skončení testu metoda vrací jednu z hodnot enumerátoru \inlinecode{LibState}. 

\subsubsection{Ukončení testování}

Po dokončení celého testovacího běhu je zavolána metoda \inlinecode{Stop()}. Ta odešle všem stále připojeným účastníkům direktivu k ukončení testování a ukončí spojení. Služba se snaží v případě chyby ukončit co nejvíce zařízení skrze stanovený protokol.


\subsubsection{Pomocné metody a třídy}
Třída \inlinecode{TestService} obsahuje navíc pomocnou metodu \inlinecode{GetParticipantIp}. Tato metoda slouží k získání IP adresy testovaného zařízení na základě znalosti MAC adresy zařízení. Pokud je zařízení s MAC adresou, kterou metoda obdrží v argumentu funkce, připojeno, tak poté metoda úkládá do výstupního argumentu jeho IP adresu a vrací úspěch. V opačném případě vrací neúspěch a jako IP adresu ukládá hodnotu \inlinecode{null}.

Třída také využívá při svém běhu dvě pomocné třídy, jejichž diagram můžeme vidět na obrázku \ref{fig:utility}. První třídou je třída \inlinecode{Logger}. Tato třída realizuje zapisování protokolu průběhu testovacího běhu do souboru. Zároveň je konstruována tak, že může zapisovat maximálně jedno vlákno programu současně. 

Třída je taktéž obsažena ve třídě \inlinecode{GlobalVar}, která simuluje globální proměnnou. Inicializaci instance třídy \inlinecode{Logger} lze provést odkudkoliv, avšak stejně jako u konfigurace pouze jen jednou. Třída \inlinecode{GlobalVar} obsahuje vlastnost \inlinecode{isLogEnabled}, která obsahuje informaci o tom, zdali je zapisování inicializováno.  

Třída \inlinecode{TestService} obsahuje ve svých metodách kontrolu, zdali je instance třídy \inlinecode{Logger} inicializována a v případě kladného vyhodnocení zapisuje průběh svého běhu. Instance třídy \inlinecode{Logger} by teda měla být vytvořena ještě před započnutím běhu služby.

Druhou třídou, kterou třída \inlinecode{TestService} využívá, je třída \inlinecode{ReportGenerator}. Tato třída vytváří výstupní XML soubor s výsledky jednotlivých testů.
Tento soubor je vytvořen ve formátu dle frameworku NUNit, jehož definici můžeme najít v \cite{nunit}. Tento soubor následně může být nahrán do serveru Azure DevOps. V navrženém řešení je tento výstup považován za záložní.


\subsection{Propojení s ostatními komponentami}

O propojení třídy \inlinecode{TestService} s ostatními komponentami knihovny se stará třída \inlinecode{ServiceRunner}.
Tato třída ve svém konstruktoru inicializuje konfiguraci služby a zjistí, zda je validní. Tento konstruktor přijímá jeden argument -- časový limit na inicializaci. V případě nezadání tohoto parametru třída využije defaultní časový limit definovaný ve třídě na 60 sekund. Následně konstruktor vytvoří instanci třídy \inlinecode{TestService} a zavolá metodu \inlinecode{Init}. V případě jakékoliv chyby je služba vyhodnocena jako by byla v chybovém stavu. Toto určuje enumerátor \inlinecode{LibState} v proměnné \inlinecode{State}. Třída využívá pouze hodnoty \inlinecode{STATE\_OK} a \inlinecode{STATE\_ERROR}. Následný důvod vyhodnocení chyby je uložen v proměnné \inlinecode{Response}. Třída má další metody, skrze které umožňuje ovládání instance třídy \inlinecode{TestService}. Jsou to tyto metody:

\begin{itemize}
    \item \inlinecode{bool RunTest<T>(T test, int timeout = DefaultTimeout)} \\ Metoda předá instrukci pro spuštění testu. Očekává stejné parametry jako metoda \inlinecode{RunTest} třídy \inlinecode{TestService}. Jedinou změnou je, že pokud časový limit nebude zadán, bude použit defaultní časový limit.
    \item \inlinecode{bool AddTestPartner(TestPartner participant)} \\ Metoda spouští testovacího partnera předaného v argumentu a předává informaci o očekávání jeho připojení  instanci třídy \inlinecode{TestService}. Instanci partnera následně uloží do seznamu partnerů.
    \item \inlinecode{void TestCleanup()} \\ Metoda je volána po dokončení jednotlivého testu, předá instanci třídy \inlinecode{TestService} direktivu k odpojení testovacích partnerů. Následně zkontroluje, zda se partneři ukončili a pokud ne, tak je ukončí.
    \item \inlinecode{void Stop()} \\ Metoda předá informaci o ukončení testovacího běhu.
    \item \inlinecode{bool GetParticipantIp(PhysicalAddress macAddr,out IPAddress \\addr)} \\
    Zpřístupnění stejnojmenné metody z třídy \inlinecode{TestService}.
\end{itemize}

\begin{figure}[H]
    \centering 
    \includegraphics[width=0.95\textwidth]{assets/img/class_diagram/utility.pdf}
    \caption{Diagram tříd, které zajišťují pomocné služby}
    \label{fig:utility}
\end{figure}


\section{Rozhraní pro testovaná zařízení}\label{sec:testrunner}

Diagram tříd, které realizují implementaci pro účastníka testování, můžeme pro implementaci v jazyku \cpp{} vidět na obrázku \ref{fig:test_client_cpp} a pro implementaci v jazyku \csharp{} na obrázku \ref{fig:test_client_csharp}. Obě implementace obsahují rozhraní pro testované zařízení, které je následně na testovaném zařízení implementováno testerem.

Vytvořená rozhraní jsou následně využita ve třídě \inlinecode{TestRunner}. Tato třída zajišťuje běh jednotlivých účastníků testování. Obsahuje tyto metody: 

\begin{itemize}
    \item \inlinecode{Init(ipAddress,port)} \\ Metoda inicializuje připojení s testovací službou, kde parametry připojení ke službě jsou obdrženy v argumentech funkce.
    \item \inlinecode{HandleInstructions()} \\ Metoda přijímá instrukce od testovací služby a na jejich základě provádí úkony.
    \item \inlinecode{RunTest(testIdentifier, testState)} \\ Metoda spouští jednotlivé fáze testů. 
    \item \inlinecode{Stop()} \\ Metoda ukončuje testovací běh.
\end{itemize}

Bližší vysvětlení si zaslouží metoda \inlinecode{RunTest}. Tato metoda je volána z metody \inlinecode{HandleInstructions}. K udržení stádia testu si třída mezi jednotlivými stádii drží instanci jednotlivých testů. Ve fázi přípravy na testování metoda zkontroluje, že je instance testu nastavena na hodnotu \inlinecode{null}. Jiná hodnota by značila chybu během předchozího testování. Ve fázi úklidu po dokončení testu je odkaz na instanci testu nastaven zpět hodnotu null. Třída \inlinecode{TestRunner} je opět implementována v jazyce \csharp{} a \cpp{}.

Návrhově je rozhraní pro zařízení i třída \inlinecode{TestRunner} v obou implementacích ekvivalentní. Je třeba ale upřesnit implementaci v \cpp{}. Tato implementace je mířena na testované zařízení, které běží na vlastním speciálně vyvinutém hardwaru. Zařízení však v některých případech nepoužívá některé vestavěné funkce a místo nich používá jejich vlastní náhradu. 

Implementace v \cpp{} tedy definuje seznam funkcí, které jsou potřeba implementovat, resp. je potřeba u nich vytvořit odkaz na funkce, které jsou implementované na zařízení. Seznam těchto funkcí můžeme vidět na výpisu \ref{listing:cppinterface}. Z něj lze vyčíst, že se jedná o funkce, které alokují a uvolňují paměť na haldě, a funkci, které paměť kopíruje. 

\begin{listing}[H]
    \centering
    \begin{minted}[breaklines, fontsize=\small]{cpp}
/*  Allocation of memory on heap */
void* TESTLIB_ALLOC_MEM(const uint32_t length);

/* Free of allocated memory */
bool TESTLIB_FREE_MEM(void* ptr);

/* Copy of memory */
void* TESTLIB_MEMCPY(void* const lpDst, const void* const lpSrc, int dwNb);
    \end{minted}
    \caption{Seznam funkcí k implementaci na zařízení v jazyce \protect\cpp{}}
    \label{listing:cppinterface}
\end{listing}


\section{Testovací partner}
Implementace testovacího partnera přímo kopíruje navržené rozhraní pro testované zařízení. Jeho logika je obsažena ve třídě \inlinecode{TestPartner}. Diagram třídy můžeme vidět také na obrázku \ref{fig:test_client_csharp}. Tato třída má tři možné konstruktory:

\begin{itemize} 
    \item \inlinecode{TestPartner(ITestClient device)} \\
    Konstruktor dostane v argumentu odkaz na instanci rozhraní zařízení.
    \item \inlinecode{TestPartner (ITestCase testCase)} \\
    Konstruktor dostane v argumentu odkaz na instanci jednoho testu, který je odvozený z rozhraní pro test.
    \item \inlinecode{TestPartner (Func<UInt32, ITestCase> getTest)} \\
    Argumentem konstruktoru je funkce, která obsahuje seznam testů. Tato funkce musí po obdržení číselné reprezentace testu v argumentu vrátit instanci testu, který je odvozený z rozhraní testu.
\end{itemize}

Třída následně vytvoří instanci implementovaného rozhraní pro zařízení (třída \inlinecode{PartnerDevice}), který přijme v konstruktoru vybraný způsob reprezentace testů. Instanci rozhraní následně předá instanci třídy \inlinecode{TestRunner} a po zavolání metody \inlinecode{Start} spustí instanci třídy \inlinecode{TestRunner} na vlastním vlákně.

Třída rovněž obsahuje metodu \inlinecode{Stop}, která po zavolání kontroluje ukončení testovacího partnera v daném časovém limitu. Po vypršení časového limitu funkce ukončuje vlákno, na kterém instance třídy \inlinecode{TestRunner} běží.

\begin{figure}[p!]
    \centering 
    \includegraphics[width=\textwidth]{assets/img/class_diagram/client-cpp.pdf}
    \caption{Diagram tříd implementace účastníka testování v jazyce \protect\cpp{}}
    \label{fig:test_client_cpp}
\end{figure}

\begin{figure}[p!]
    \centering 
    \includegraphics[width=\textwidth]{assets/img/class_diagram/client-csharp.pdf}
    \caption{Diagram tříd implementace účastníka testování v jazyce \protect\csharp{}}
    \label{fig:test_client_csharp}
\end{figure}

\clearpage


\section{Propojení s frameworkem MSTest}

Podstatnou součástí implementace je propojení služby s testovacím frameworkem MSTest. O toto se stará třída \inlinecode{API}, jejíž diagram můžeme vidět na obrázku \ref{fig:api}. Jako jediná využívá nástroje tohoto frameworku. Je koncipována jako statická třída, i když je závislá na jím vytvořené instanci třídy \inlinecode{ServiceRunner}. Toto je utvořeno kvůli kompatibilitě s frameworkem MSTest. 

\begin{figure}[H]
    \centering 
    \includegraphics[width=0.6\textwidth]{assets/img/class_diagram/api.pdf}
    \caption{Diagram třídy API}
    \label{fig:api}
\end{figure}


\subsection{Použití frameworku}
Jak jsem již zmínil v sekci \ref{sec:reg_test_design}, framework MSTest pro identifikaci testových metod využívá atributy, například:

\begin{itemize}
    \item \inlinecode{AssemblyInitialize} -- identifikuje inicializační funkci, která je spuštěna před spuštěním testů \cite{attr_init_clean}.
    \item \inlinecode{AssemblyCleanup} -- identifikuje funkci, která bude spuštěna po skončení všech testů \cite{attr_init_clean}.
    \item \inlinecode{TestClass} -- identifikuje třídu, která obsahuje testy \cite{mstest_docs}.
    \item \inlinecode{TestMethod} -- identifikuje metody, které reprezentují jednotlivé testy \cite{mstest_docs}.
\end{itemize}

Framework MSTest následně používá tyto metody, které jsou identifikovány těmito a dalšími atributy. Při kompilaci jazyk \csharp{} vytváří jednotlivé celky kompilovaného kódu, které poté spojuje do logického celku. Tyto jednotlivé celky se nazývají \textit{assembly}. \cite{assembly}

Implementace testovací knihovny a uživatelské použití knihovny, a tedy i testovacího frameworku MSTest, tvoří samostatné logické celky. Framework MSTest následně detekuje správně definované funkce pouze z toho logického celku, ze kterého je spouštěn.

\subsection{Spuštění a ukončení služby}

Při spuštění jednotlivých testů je nutné brát v potaz, že testovací služba neví, jaké testy budou kdy spuštěny. Testovací služba se tedy musí používat nezávisle na všech testech. K tomuto využijeme dvě funkce frameworku MSTest~-- \inlinecode{AssemblyInitialize} a \inlinecode{AssemblyCleanup}. Tyto funkce jsou spuštěny před započetím a po skončení testovaní. Ve třídě \inlinecode{API} jejím implementacím odpovídají metody \inlinecode{AssemblyInit} a \inlinecode{AssemblyCleanup}. Metoda \inlinecode{AssemblyInit} přijímá dva argumenty:

\begin{itemize}
    \item \inlinecode{Assembly assembly} -- odkaz na runtime blok, který odkazuje na uživatelskou část použití testovací knihovny.
    \item \inlinecode{int InitTimeout} -- časový limit na inicializační fázi služby (tedy časový limit, který je využíván v metodě \inlinecode{TestService.Init}), pokud je ponechán prázdný, je použit defaultní časový limit z třídy \inlinecode{ServiceRunner}.
\end{itemize}

Metoda \inlinecode{AssemblyInit} na počátku kontroluje kolize mezi testovými identifikátory. Způsob kontroly je popsán v sekci \ref{sec:reg_test_impl}. Metoda následně inicializuje službu, která projde inicializační fází. 

Metoda \inlinecode{AssemblyCleanup} symetricky ukončuje běh testovací služby. V určitých případech tato funkce nemusí být zavolána. Při zjištění chyby v běhu, která vyústí v ukončení testovacího běhu, je tedy tato funkce zavolána manuálně. Zároveň je kontrolováno, aby tato metoda nebyla volána vícekrát.

Aby mohl MSTest rozeznat tyto funkce, je potřeba je definovat ve stejné assembly, která spouští tento framework. Toto vytváří problém u automatizace vytvoření propojení s testovací knihovnou. Pokud v testovací knihovně definujeme metody, které budou obsahovat atributy \inlinecode{AssemblyInitialize} a \inlinecode{AssemblyCleanup}, testovací framework  MSTest tyto metody nevidí, a tak je nepoužije. Proto je potřeba vytvořit dvě funkce, které budou obalovat metody \inlinecode{AssemblyInit} a \inlinecode{AssemblyCleanup}, ve stejné assembly jako ty, ve které běží framework MSTest.

Toto spravuje statická třída \inlinecode{TestLibInit},která pouze zaobaluje tyto dvě metody svými metodami a přidává k nim potřebné atributy. Třída není součástí assembly testovací knihovny. V kontextu samotné knihovny je to pouze textový soubor. Proces propojení této třídy s testovacím projektem, ve kterém poběží framework MSTest, je popsán v sekci \ref{sec:distrbution}.

\subsection{Registrování testů}\label{sec:reg_test_impl}

Při rozlišování jednotlivých testů knihovna a testovací služba využívá enumerátory, jenž odkazují na nějakou číselnou hodnotu, která je následně odesílána účastníkům testování. Všechny enumerátory musí být označeny atributem \inlinecode{TestEnum}. Díky němu může knihovna identifikovat výčtové typy, které reprezentují testy a zjistit kolize mezi nimi. 

Tato kontrola je provedena před započetím testování v metodě \inlinecode{AssemblyInit} díky odkazu na runtime blok, který obdrží v argumentu. Tato kontrola se dá přeskočit předáním hodnoty \inlinecode{null} jako odkazu na runtime blok v argumentu metody.

Framework MSTest registruje testy nezávisle na testovací knihovně. Jednotlivé třídy musí být označeny atributem \inlinecode{TestClass} a zároveň jednotlivé metody atributem \inlinecode{TestMethod}. Framework následně testy detekuje automaticky.

\subsection{Spuštění testu}
Pro spuštění testu je použita metoda \inlinecode{Run} z třídy \inlinecode{API}. Tato metoda přijímá stejné argumenty jako metoda \inlinecode{RunTest} třídy \inlinecode{ServiceRunner}. Metoda zkontroluje, že argument identifikující test je typem enumerátor a obsahuje atribut \inlinecode{TestEnum}. V případě, že se testovací služba nenachází v chybném stavu, předá metoda testovací službě direktivu k započetí testu. Následně vyhodnotí jeho úspěšnost. Je-li služba je v chybném stavu, metoda vyhodnotí test jako bezvýsledný.

Zároveň lze před započetím testu přidat testovací partnery. Toto je možné za pomocí metody \inlinecode{AddTestPartner} z třídy \inlinecode{API}. Metoda v argumentu obdrží instanci testovacího partnera a následně se postará o jeho běh.

\subsection{Pomocné metody}
Jedním z návrhových cílů třídy \inlinecode{API} je, aby všechny uživatelsky podporované operace šli provést skrz tuto třídu. Třída proto obsahuje tyto pomocné metody:
\begin{itemize}
    \item \inlinecode{static void EnableLog(string fileName = "log.txt")}\\
    Metoda inicializuje zápis běhu testovací knihovny, který je realizovaný třídou \inlinecode{Logger}. Název výstupního souboru je předán v argumentu, případně je použit defaultní název.
    \item \inlinecode{static bool GetParticipantIp(PhysicalAddress macAddr, out \\ IPAddress addr)} \\
    Zpřístupnění stejnojmenné metody z třídy \inlinecode{ServiceRunner}.
\end{itemize}


\section{Vytvoření balíčku NuGeT a distribuce knihovny}\label{sec:distrbution}

Knihovna je distribuována jako NuGet balíček. Vytvořený balíček NuGet je následně nahrán do služby Azure Artifacts. K jeho vytvoření stačí definovat konfigurační soubor s příponou \inlinecode{.nuspec}, který definuje metadata balíčku a závislosti na ostatních NuGet balíčcích. Balíček je následně vytvořen za pomocí příkazu \inlinecode{nuget pack}. 

NuGet balíček umožňuje i distribuci souborů, které nejsou součástí kompilované knihovny. Tuto funkci využijeme k distribuci několika souborů. Balíček obsahuje informační soubor, který popisuje základní informace k použití balíčku. Taktéž ale obsahuje soubory, které usnadňují použití knihovny. 

Prvním je \inlinecode{TestLibInit.cs}. Tento soubor obsahuje třídu \inlinecode{TestLibInit}, která obaluje metody \inlinecode{AssemblyInit} a \inlinecode{AssemblyCleanup} třídy \inlinecode{API}. Soubor je díky NuGet balíčku automaticky vytvořen při instalaci a zároveň je automaticky přidán mezi soubory, které jsou v testovacím projektu kompilovány. 

Dalším souborem je soubor \inlinecode{TestEnumTemplate.cs.txt}. Tento soubor obsahuje šablonu souboru, který bude obsahovat enumerátory, které identifikují testy. Tato šablona obsahuje preprocesorové direktivy, které dělají tento soubor kompatibilní jak pro jazyk \csharp{}, tak pro jazyk \cpp{}. Soubor má přidanou příponu \inlinecode{.txt} kvůli tomu, aby nebyl automaticky přidán do kompilace testovacího projektu.

Posledním souborem je soubor \inlinecode{config.xml}. Tento soubor obsahuje nastavení služby, které již bylo definováno v sekci \ref{sec:settings}. Všechny tyto soubory jsou po instalaci uloženy do složky \inlinecode{resources}.

Je dobré vytvořit kopii souborů \inlinecode{TestEnumTemplate.cs.txt} a \inlinecode{config.xml} a přesunout je mimo složku \inlinecode{resources}. Při aktualizaci NuGet balíčku jsou totiž aktualizovány všechny soubory, které jsou na něj vázané. Toto může zapříčinit přepsání konfiguračních souborů, pokud nebudou přesunuty. 

