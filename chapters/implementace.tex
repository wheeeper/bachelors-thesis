\chapter{Implementace}

Tato kapitola se věnuje implementační části všech navrhnutých komponent, které umožnují automatizaci testování.

\section{Komunikace}
Testovací knihovna má jasně definovanou komunikaci. Strukturu jedné zprávy zajišťuje třída \inlinecode{Message}. Dle návrhu má jedna odeslaná zpráva definované tři atributy -- délku zprávy, typ zpráva a data zprávy. Třída \inlinecode{Message} si drží dva z těchto atributů, třetí -- délka zprávy -- je na základě ostatních dvou vypočítána. Typ zprávy je určen enumerátorem \inlinecode{MessageType}. Tento enumerátor obsahuje výčty:

\begin{itemize} 
    \item \inlinecode{MSG\_FAIL} -- zpráva o neúspěchu
    \item \inlinecode{MSG\_OK} -- zpráva o úspěchu nebo potvrzení
    \item \inlinecode{MSG\_RUNTEST} -- pokyn k započnutí testu
    \item \inlinecode{MSG\_STOP} -- pokyn k ukončení testování
    \item \inlinecode{MSG\_EXCEPTION} -- chybná zpráva, slouží pro rozpoznání neplatné zprávy
\end{itemize}

Data zprávy jsou uložena v dynamickém poli. 

\todo{Doplnit}


\section{Testovací služba}
Jak již bylo zmíněno, jádrem k ovládání testování je testovací služba. V implementaci je tento celek rozdělen na dvě hlavní třídy. Třída \inlinecode{TestService} obstarává jednotlivé úkony služby, následně třída \inlinecode{ServiceRunner} propojuje třídu \inlinecode{TestService} s ostatními komponentami, které jsou potřebné pro testování.

\subsection{Nastavení služby}

Konfigurace služby je uložena v souboru \inlinecode{config.xml}. Tento soubor obsahuje tři hodnoty pro nastavení:

\begin{itemize}
    \item \inlinecode{ip} -- Adresa, na které bude služba poslouchat příchozí připojení. Všechna zařízení se budou připojovat na tuto adresu.
    \item \inlinecode{port} -- Síťový port, skrz který služba provádí komunikaci
    \item \inlinecode{participants} -- Počet fyzických participantů, které se připojí ke službě.   
\end{itemize}

Ukázku tohoto souboru můžeme vidět na výpisu \ref{listing:configxml}. V knihovně se o konfiguraci stará třída \inlinecode{Configuration}. Tato třída existuje v jmenném prostoru \inlinecode{GlobalVar}. Tento jmenný prostor simuluje globální proměnnou. Kvůli tomu jsou ale všechny třídy, které jsou odsud používány, konstruovány tak, aby primárně fungovaly jen ke čtení. Při vytvoření instance třída získá konfiguraci z konfiguračního souboru a uloží je do vnitřních proměnných třídy. Tyto proměnné jsou určeny pouze ke čtení, nelze je upravovat. Stav konfigurace určuje proměnná \inlinecode{bool IsValid}. V případě chyby je tato proměnná nastavená na hodnotu \inlinecode{false} a v proměnné \inlinecode{Exception} je uložen důvod neúspěchu. V opačném případě je tato proměnná nastavena na hodnotu \inlinecode{true}.

\begin{listing}[htbp]
    \centering
    \begin{minted}{xml}
        <?xml version="1.0" encoding="UTF-8" ?>
        <configuration>
          <!-- IP of the service-->
          <ip>192.168.4.100</ip>
          <!-- Port of the service -->
          <port>1337</port>
          <!-- Number of non-virtualized participants-->
          <participants>1</participants>
        </configuration>
    \end{minted}
    \caption{Ukázka konfiguračního souboru}
    \label{listing:configxml}
\end{listing}


\subsection{Operace služby}

Jak jsem již zmínil, jednotlivé úkony služby jsou implementovány v třídě \inlinecode{TestService}. Třída po své konstrukci inicializuje vnitřní proměnné, ale neprovádí žádné úkony. 

\subsubsection{Odesílání a přijímání zpráv}
Struktura jedné zprávy je již známým faktem. O jednotlivá připojení se avšak stará třída \inlinecode{TestService}. Třída má dvě hlavní statické metody pro odesílání a přijímání jednotlivých zpráv. Tyto metody jsou:

\begin{itemize}
    \item \inlinecode{static void SendMessage(TcpClient client, Message msg)} \\
    Metoda pro odeslání zprávy jednomu klientovi
    \item \inlinecode{static bool RcvMessage(TcpClient client, out Message msg)} \\
    Metoda pro přijmutí zprávy od jednoho klienta
\end{itemize}

Všechny zprávy jsou následně odesílány a přijímány za pomoci těchto dvou metod. Rozšířením těchto metod jsou metody \inlinecode{SendParticipantsMessage} a \inlinecode{CheckParticipantsResponse}. Obě metody pracují se všemi participanty připojenými ke službě. Metoda \inlinecode{SendParticipantsMessage} odešle všem participantům jednu zprávu, naopak metoda \inlinecode{CheckParticipantsResponse} zprávu od všech participantů přijme a zkontroluje, zda se všechny typy zprávy shodují s očekávaným typem zprávy. Očekávaný typ zprávy je předán v argumentu metody. \todo{CheckParticipantsResponse co vrací}

\subsubsection{Inicializace}

Metodou \inlinecode{bool Init(int InitTimeout)} třída \inlinecode{TestService} inicializuje běh služby. Proměnná \inlinecode{InitTimeout} určuje dobu, kdy služba čeká na příchozí komunikaci. Doba je metodě předávána, stejně jako všem ostatním metodám, které mají definovaný časový limit, v milisekundách. 

Metoda z konfigurace zjistí IP adresu a port, na kterém má služba běžet. Následně začne na této IP adrese a portu poslouchat příchozí komunikaci. V této fázi se připojují pouze fyzická zařízení. Z nastavení služba ví, kolik participantů má očekávat. 

V případě příchozí komunikace je poté zavolána metoda \inlinecode{AddClient}. Tato metoda vytvoří klienta a přijme identifikační zprávu od participanta. Tato zpráva má typ zprávy \inlinecode{MSG\_OK} a v datech zprávy obsahuje MAC adresu participanta. Metoda odešle participantovi potvrzovací zprávu s typem zprávy \inlinecode{MSG\_OK}, bez dat. 

Po úspěšném ověření připojení metoda uloží vytvořeného klienta a jeho MAC adresu do seznamu připojených participantů. Metoda zároveň rozezná připojení virtuálního participanta. Tento participant má MAC adresu \inlinecode{DE:AD:BE:EF:00:00}. Nakonec metoda vrací tři hodnoty typu integer:
\begin{itemize}
    \item \inlinecode{0} -- pokud připojování participanta vyústilo v neúspěch, ať už kvůli překročení časového limitu, nebo kvůli nedodržení stanovené komunikace.
    \item \inlinecode{1} -- pokud služba úspěšně přijme fyzického participanta
    \item \inlinecode{2} -- pokud služba úspěšně přijme virtuálního participanta
\end{itemize}

Služba úspěšně ukončuje inicializační fázi poté co se úspěšně připojí stejný počet participantů, jako bylo určeno v konfiguraci. V případě nepřipojení se očekávaného počtu zařízení v definovaném čase, obdržení špatné nebo žádné zprávy služba vyhodnocuje inicializační fázi jako neúspěšnou. Připojení virtualizovaného participanta v této fázi vyústí též v neúspěch inicializace.

\subsubsection{Spuštění testu}

Nejpodstatnější operací je spuštění jednotlivých testů. O toto se stará metoda \inlinecode{TestResult RunTest<T>(T testEnum, int timeout)}. Metoda očekává dva parametry:

\begin{itemize}
    \item \inlinecode{T testEnum} -- Identifikátor testu s generickým identifikátorem T, který typem musí být enumerátor.
    \item \inlinecode{int timeout} -- Maximální délka, po kterou služba očekává odpověď od participantů testu. 
\end{itemize}

Pro započnutí testu na testovaném zařízení služba odešle zprávu, s typem zprávy \inlinecode{MSG\_RUNTEST}. Data této zprávy obsahují:

\begin{enumerate}
    \item Číselnou reprezentaci enumerátoru \inlinecode{testEnum}. Ten je uložen v prvních 4 bytech dat zprávy.
    \item Číselnou reprezentaci enumerátoru \inlinecode{TestStateE}, určující stádium testování. Uložen je ve dvou bajtech dat zprávy hned za identifikátorem testu.
\end{enumerate}

Tato metoda postupně spouští všechny stádia testování. Tyto stádia jsou rozlišovány za pomocí emulátoru \inlinecode{TestStateE}. Jednotlivě:

\begin{itemize}
    \item \inlinecode{TEST\_STARTUP} -- příprava na testování
    \item \inlinecode{TEST\_RUN} -- spuštění testu
    \item \inlinecode{TEST\_TEARDOW} -- úklid po testu
\end{itemize}

Po spuštění jednotlivých stádií testování metoda čeká na odpověď od všech participantů. Participant odesílá službě zprávu s typem \inlinecode{MSG\_OK} pro úspěch, a nebo s typem \inlinecode{MSG\_FAIL} pro neúspěch. Obě zprávy neobsahují žádné data.

Doba čekání je určena argumentem \inlinecode{timeout}. Tento čas je vázaný na jednotlivá stádia testování. Tedy pokud metoda má časový limit 30 sekund, tak poté bude po každém stádiu testování čekat 30 sekund na odpověď od participanta. V případě že vyprší časový limit, tak je participant považován jako v chybném stavu. Taktéž je participant považován jako v chybném stavu, pokud vrátí neúspěch u přípravy na testování nebo u úklidu po testu.

Metoda následně vrací jednotu z hodnot enumerátoru \inlinecode{TestResult}. Tyto hodnoty jsou:

\begin{itemize}
    \item \inlinecode{TEST\_SUCCESS} -- test proběhl úspěšně
    \item \inlinecode{TEST\_FAIL} -- test proběhl neúspěšně
    \item \inlinecode{TEST\_ERROR} -- v průběhu testování se vyskytla chyba a další testování není možné   
\end{itemize}


\subsubsection{Ukončení testování}

Po dokončení celého testovacího běhu je zavolána metoda \inlinecode{Stop()}. Tato metoda odešle všem participantům stále připojeným ke službě zprávu s typem \inlinecode{MSG\_STOP} a ukončí spojení. Testování avšak může být ukončeno i v případě vzniklé chyby. Služba se snaží v případě chyby ukončit co nejvíce zařízení skrz stanovený protokol.

\subsubsection{Virtuální participanti}

Jelikož každý test může obsahovat různý počet virtuálních participantů, je podstatné, aby služba mohla při testovacím běhu tyto participanty přidávat a odebírat. K tomuto slouží dvě metody. Metoda \inlinecode{AddVirtualParticipant} řekne službě, že má očekávat připojení virtuálního participanta. V případě úspěšného připojení virtuálního participanta služba vrací úspěch. Naopak případě chyby, nebo připojení fyzického participanta, metoda vrací neúspěch. O opačnou operaci se stará metoda \inlinecode{StopVirtualDevices}. Metoda po svém zavolání odešle všem virtuálním participantům zprávu o ukončení testování a tato zařízení jsou odpojena a ukončena.

\subsection{Propojení s ostatními komponentami}

O propojení třídy \inlinecode{TestService} s ostatními komponentami knihovny se stará třída \inlinecode{ServiceRunner}.
Tato třída ve svém konstruktoru inicializuje konfiguraci služby a zjistí, zda je validní. Tento konstruktor přijímá jeden argument -- časový limit na inicializaci. V případě nezadání tohoto parametru třída využije defaultní časový limit definovaný ve třídě na 60 sekund. Následně konstruktor vytvoří instanci třídy \inlinecode{TestService} a zavolá metodu \inlinecode{Init}. V případě jakékoliv chyby je služba vyhodnocena jako v chybovém stavu. Toto určuje enumerátor \inlinecode{RunnerState}, který má dvě možné hodnoty: 

\begin{itemize}
    \item \inlinecode{ERROR} -- služba je v chybovém stavu
    \item \inlinecode{OK}  -- služba je v pořádku  
\end{itemize}

Tato informace je uložena ve vlastní proměnné \inlinecode{State}. Následný důvod vyhodnocení chyby je uloženo v proměnné \inlinecode{Exception}. Třída následně má další metody, skrz které umožňuje ovládání instance třídy \inlinecode{TestService}. Tyto metody jsou:

\begin{itemize}
    \item \inlinecode{bool RunTest<T>(T test, int timeout = DefaultTimeout)} \\ Metoda předá instrukci pro spuštění testu. Očekává stejné parametry jako metoda \inlinecode{RunTest} třídy \inlinecode{TestService}. Jedinou změnou je, že pokud časový limit nebude zadán, tak bude použit defaultní časový limit.
    \item \inlinecode{bool AddVirtualParticipant()} \\ Metoda předá informaci o očekávání připojení virtuálního participanta.
    \item \inlinecode{void TestCleanup()} \\ Metoda je volána po dokončení jednotlivého testu. V aktuální implementaci třída předá informaci o požadavku na odpojení virtuálních participantů.
    \item \inlinecode{void Stop()} \\ Metoda předá informaci o ukončení testovacího běhu.
\end{itemize}


\section{Rozhraní pro testovaná zařízení}

Rozhraní pro testovaná zařízení umožňuje propojení s testovací službou. Jeho cílem je být co nejjednodušší pro co nejrychlejší implementaci. Definici rozhraní pro jazyk \inlinecode{C++} můžeme vidět na výpisu \ref{listing:dev_if}.

\begin{listing}[htbp]
    \begin{minted}[breaklines]{cpp}
    class DeviceIf
    {
    public:

        virtual ~DeviceIf() {}

        virtual bool createConnection(const char * ipAddress, uint16_t port) = 0;

        virtual bool sendMessage(const Message& msg) = 0;

        virtual bool rcvMessage(Message& msg) = 0;

        virtual TestCaseIf * getTest(uint32_t test) = 0;

        virtual bool getMacAddress(uint8_t** macAddr, uint32_t* size) = 0;
    };
    \end{minted}
\caption{Ukázka definice rozhraní}
\label{listing:dev_if}
\end{listing}

\todo{Dopsat}


\section{Virtuální participant}
Implementace virtuálního participanta přímo kopíruje navrhnuté rozhraní pro testované zařízení. Logika virtuálního participanta je obsažena ve třídě \inlinecode{VirtualClient}. Tato třída tři možné konstruktory:

\begin{itemize} 
    \item \inlinecode{VirtualClient(ITestClient device)} \\
    Konstruktor dostane v argumentu odkaz na implementaci rozhraní zařízení
    \item \inlinecode{VirtualClient (ITestCase testCase)} \\
    Konstruktor dostane v argumentu odkaz na implementaci jednoho testu, který je odvozený z rozhraní pro test.
    \item \inlinecode{VirtualClient (Func<UInt32, ITestCase> getTest)} \\
    Argumentem konstruktoru je funkce, která obsahuje seznam testů. Funkce v argumentu dostane číselnou reprezentaci testu a vrací instanci testu, který je odvozený z rozhraní testu.
\end{itemize}

Třída následně vytvoří instanci implementace rozhraní klienta. Tato implementace je obsažena ve třídě \inlinecode{TestClientDefault}. Testovací běh virtuálního participanta funguje na stejném principu jako běh ostatních testovaných zařízení. Rozdílem je, že třída \inlinecode{VirtualClient} vytvoří nové vlákno, na kterém běží tento virtuální participant. Vytvořené vlákno (a s ním i virtuální participant) je poté spuštěno metodou \inlinecode{Start}. Následně metoda \inlinecode{Stop} ukončuje běh virtuálního participanta. Metoda nejdříve čeká, zda se participant a vlákno ukončí standardním způsobem. Pokud se virtuální participant neukončí ve stanoveném časovém limitu, tak ho metoda ukončí ukončením běžícího vlákna.
\todo{Doplnit}


\section{Propojení s frameworkem MSTest}

Podstatnou součástí implementace je propojení služby s testovacím frameworkem MSTest. O toto se stará třída \inlinecode{API}. Jako jediná třída využívá nástroje tohoto frameworku. Třída je koncipovaná jako statická třída, i když je závislá na jím vytvořené instanci třídy \inlinecode{ServiceRunner}. Toto je uděláno z důvodu designového omezení frameworku MSTest. 

Při spuštění jednotlivých testů je nutné brát v potaz, že testovací služba neví jaké testy budou kdy spuštěny. Testovací služba se tedy musí používat nezávisle na těchto testech. K tomuto využijeme dvě funkce frameworku MSTest -- \inlinecode{AssemblyInitialize} a \inlinecode{AssemblyCleanup}. Tyto dvě funkce jsou spuštěny ještě před započnutím a po skončení testovaní. V třídě \inlinecode{API} jejím implementacím odpovídají metody \inlinecode{AssemblyInit} a \inlinecode{AssemblyCleanup}. Metoda \inlinecode{AssemblyInit} přijímá dva argumenty:

\begin{itemize}
    \item \inlinecode{Assembly assembly} -- odkaz na runtime blok, který reprezentuje spuštěný blok programu mimo knihovnu
    \item \inlinecode{int InitTimeout} -- časový limit na inicializační fázi služby (tedy časový limit, který je využíván v metodě \inlinecode{TestService.Init}), pokud je ponechán prázdný, je použit defaultní časový limit z třídy \inlinecode{ServiceRunner}.
\end{itemize}

Předání runtime bloku -- neboli assembly -- je sice dobrovolným krokem, ale podstatným. Při odesílání testů služba odesílá pouze číselnou reprezentaci enumerátoru, který test identifikuje. Tento enumerátor ale nemusí být stejného typu. Proto zde může vzniknout kolize.

Framework MSTest využívá pro rozlišení testovacích součástí tzv. atributy. Atributy přidávají metadata k třídám, typům, metodám atd. \cite{attribute_docs}. Framework MSTest využívá například tyto atributy:

\begin{itemize}
    \item \inlinecode{AssemblyInitialize} -- identifikuje inicializační funkci
    \item \inlinecode{AssemblyCleanup} -- identifikuje funkci která bude spuštěna po skončení všech testů
    \item \inlinecode{TestClass} -- identifikuje třídu, která obsahuje testy
    \item \inlinecode{TestMethod} -- identifikuje metody, které reprezentují jednotlivé testy
\end{itemize}

Testovací knihovna tuto sadu atributů rozšiřuje o atribut \inlinecode{TestEnum}. Tento atribut říká, že enumerátor s atributem \inlinecode{TestEnum} reprezentuje nějaké testy. Proto metoda \inlinecode{AssemblyInit} prostřednictvím argumentu \inlinecode{assembly} zjistí, které enumerátory mají tento atribut. Následně zkontroluje, zdali tyto enumerátory nemají mezi sebou kolize. V případě nalezení kolize je vyhozena výjimka. Tato kontrola se dá přeskočit předáním hodnoty \inlinecode{null} v argumentu \inlinecode{assembly}.

Metoda \inlinecode{AssemblyCleanup} symetricky ukončuje běh testovací služby. V určitých případech tato funkce nemusí být zavolána \question{Je zde potřeba citace}. Při zjištění chyby v běhu, která vyústí v ukončení testovacího běhu, je tato funkce volána manuálně.

Pro spuštění testování je použita metoda \inlinecode{Run}. Tato metoda přijímá stejné argumenty jako metoda \inlinecode{RunTest} třídy \inlinecode{ServiceRunner}. Metoda zkontroluje, že argument identifikující test je typem enumerátor a obsahuje atribut \inlinecode{TestEnum}. V případě že se testovací služba nenachází v chybném stavu, tak metoda předá testovací službě direktivu k započnutí testu. Následně vyhodnotí úspěšnost testu. V případě, že služba je v chybném stavu, metoda vyhodnotí test jako bezvýsledný. 
 