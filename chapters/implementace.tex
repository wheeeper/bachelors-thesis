\chapter{Implementace}

Tato kapitola se věnuje implementaci všech navrhnutých komponent, které umožnují automatizaci testování.

\section{Zpráva}
Strukturu jedné zprávy zajišťuje třída \inlinecode{Message}. Dle návrhu instance této třídy obsahuje typ zprávy a data zprávy, pokud zpráva nějaké obsahuje. Typ zprávy je určen enumerátorem \inlinecode{MessageType}. 

Při odesílání a přijímání zpráv skrz protokol TCP/IP je podstatné, aby se instance třídy dala převádět na bajtové pole a obráceně. Třída \inlinecode{Message} proto obsahuje dvě metody, které tyto převody zajišťují. Tyto metody jsou:

\begin{itemize}
    \item \inlinecode{GetByteStream} -- Metoda převádí instanci třídy \inlinecode{Message} na bajtové pole
    \item \inlinecode{GetMessageFromStream} -- Statická metoda, které v argumentu obdrží bajtové pole a vrátí instanci třídy \inlinecode{Message}.
\end{itemize}

Následná komunikace skrz protokol TCP/IP je implementována v závislosti na zařízení, za pomocí těchto metod. Třída \inlinecode{Message} je implementována v jazyce \csharp{} a \cpp{}. Jejich implementace má minimální odlišnosti, které jsou primárně zapříčiněny rozdíly těchto dvou jazyků.

\section{Testovací služba}
Testovací služba je v implementaci rozdělena na dvě hlavní třídy. Třída \inlinecode{TestService} obstarává jednotlivé úkony služby, následně třída \inlinecode{ServiceRunner} propojuje třídu \inlinecode{TestService} s ostatními komponentami, které jsou potřebné pro testování.

\subsection{Nastavení služby}

Konfigurace služby je uložena v souboru \inlinecode{config.xml}. Tento soubor obsahuje tři hodnoty pro nastavení:

\begin{itemize}
    \item \inlinecode{ip} -- Adresa, na které bude služba poslouchat příchozí připojení. Všechna zařízení se budou připojovat na tuto adresu.
    \item \inlinecode{port} -- Síťový port, skrz který služba provádí komunikaci.
    \item \inlinecode{participants} -- Počet testovaných zařízení (tedy ne virtuálních), které se připojí ke službě.   
\end{itemize}

Ukázku tohoto souboru můžeme vidět na výpisu \ref{listing:configxml}. V knihovně se o konfiguraci stará třída \inlinecode{Configuration}. Tato třída existuje v jmenném prostoru \inlinecode{GlobalVar}. Tento jmenný prostor simuluje globální proměnnou. Kvůli tomu jsou ale všechny třídy, které jsou odsud používány, konstruovány tak, aby primárně fungovaly jen ke čtení. Při vytvoření instance třída získá konfiguraci z konfiguračního souboru a uloží je do vnitřních proměnných třídy. Tyto proměnné jsou určeny pouze ke čtení, nelze je upravovat. Stav konfigurace určuje proměnná \inlinecode{bool IsValid}. V případě chyby je tato proměnná nastavená na hodnotu \inlinecode{false} a v proměnné \inlinecode{Exception} je uložen důvod neúspěchu. V opačném případě je tato proměnná nastavena na hodnotu \inlinecode{true}.

\begin{listing}[htbp]
    \centering
    \begin{minted}{xml}
        <?xml version="1.0" encoding="UTF-8" ?>
        <configuration>
          <!-- IP of the service-->
          <ip>192.168.4.100</ip>
          <!-- Port of the service -->
          <port>1337</port>
          <!-- Number of non-virtualized participants-->
          <participants>1</participants>
        </configuration>
    \end{minted}
    \caption{Ukázka konfiguračního souboru}
    \label{listing:configxml}
\end{listing}


\subsection{Operace služby}

Jak jsem již zmínil, jednotlivé úkony služby jsou implementovány v třídě \inlinecode{TestService}. 
Třída po své konstrukci inicializuje vnitřní proměnné, ale neprovádí žádné úkony. V následujících sekcích přiblížím jednotlivé dostupné operace.

\subsubsection{Odesílání a přijímání zpráv}
Třída \inlinecode{TestService} vytváří během svého běhu propojení s účastníky testování. Po úspěšném připojení služba používá vytvořeného TCP klienta pro komunikaci. K této komunikaci má vytvořené dvě statické metody. Tyto metody jsou:

\begin{itemize}
    \item \inlinecode{static void SendMessage(TcpClient client, Message msg)} \\
    Metoda pro odeslání jedné zprávy jednomu klientovi.
    \item \inlinecode{static bool RcvMessage(TcpClient client, out Message msg)} \\
    Metoda pro přijmutí zprávy od jednoho klienta.
\end{itemize}

Všechny zprávy jsou následně odesílány a přijímány za pomoci těchto dvou metod. Rozšířením těchto metod jsou metody \inlinecode{SendParticipantsMessage} a \inlinecode{CheckParticipantsResponse}. Obě metody pracují se všemi účastníky testování připojenými, kteří jsou připojený ke službě. Metoda \inlinecode{SendParticipantsMessage} odešle všem účastníkům jednu stejnou zprávu, naopak metoda \inlinecode{CheckParticipantsResponse} zprávu od všech účastníků přijme a zkontroluje, zda se všechny typy zprávy shodují s očekávaným typem zprávy. Očekávaný typ zprávy je předán v argumentu metody. 

Metody následně v návratové hodně vrací dvě položky. První položkou je výsledek operace. Tento výsledek je reprezentován enumerátorem \inlinecode{LibState}. Tento enumerátor má tyto výčty:

\begin{itemize}
    \item \inlinecode{STATE\_OK} - označuje úspěch
    \item \inlinecode{STATE\_FAIL} - označuje neúspěch
    \item \inlinecode{STATE\_ERROR} - označuje fatální neúspěch, jehož důvodem je nějaká chyba
\end{itemize}

Tento enumerátor je zároveň hojně využíván skrz knihovnu k reprezentaci stavu různých komponent. Druhou položkou, kterou metoda vrací, je seznam zařízení, které se při komunikaci způsobily neúspěch. 
Tento neúspěch může vzniknout ať už obdržením jiné zprávy než očekávané v případě metody \inlinecode{CheckParticipantsResponse}, a nebo vznikem nějaké chyby při komunikaci.

\subsubsection{Inicializace}

Metodou \inlinecode{bool Init(int InitTimeout)} třída \inlinecode{TestService} inicializuje běh služby. Proměnná \inlinecode{InitTimeout} určuje dobu, kdy služba čeká na příchozí komunikaci. Doba je metodě předávána, stejně jako všem ostatním metodám, které mají definovaný časový limit, v milisekundách. 

Metoda z konfigurace zjistí IP adresu a port, na kterém má služba běžet. Následně začne na této IP adrese a portu poslouchat příchozí komunikaci. Z nastavení služba ví, kolik připojení má očekávat. 

V případě příchozí komunikace je poté zavolána metoda \inlinecode{AddClient}. Tato metoda vytvoří připojení s testovacím zařízením a přijme identifikační zprávu od testovaného zařízení. Po úspěšném ověření připojení metoda uloží vytvořeného klienta a jeho MAC adresu do seznamu připojených účastníků testování. Nakonec metoda vrací tři hodnoty typu integer:
\begin{itemize}
    \item \inlinecode{0} -- pokud připojení účastníka testování vyústilo v neúspěch, ať už kvůli překročení časového limitu, nebo kvůli nedodržení stanovené komunikace
    \item \inlinecode{1} -- pokud služba úspěšně přijme testované zařízení
    \item \inlinecode{2} -- pokud služba úspěšně přijme virtuálního zařízení
\end{itemize}

Služba úspěšně ukončuje inicializační fázi poté co se úspěšně připojí stejný počet účastníků testovaní, jako bylo určeno v konfiguraci. V případě nepřipojení se očekávaného počtu zařízení v definovaném čase, obdržení špatné nebo žádné zprávy služba vyhodnocuje inicializační fázi jako neúspěšnou. Připojení virtuálního zařízení v této fázi vyústí též v neúspěch inicializace.

\subsubsection{Připojení virtuálního zařízení}

Jelikož každý test může obsahovat různý počet virtuálních zařízení, je podstatné, aby služba mohla při testovacím běhu tyto zařízení přidávat a odebírat. K tomuto slouží dvě metody. Metoda \inlinecode{AddVirtualParticipant} řekne službě, že má očekávat připojení virtuálního zařízení. V případě úspěšného připojení virtuálního zařízení služba vrací úspěch. Naopak případě chyby, nebo připojení jiného zařízení, metoda vrací neúspěch. O opačnou operaci se stará metoda \inlinecode{StopVirtualDevices}. Metoda po svém zavolání odešle všem virtuálním zařízení zprávu o ukončení testování a tato zařízení jsou odpojena a ukončena.

\subsubsection{Spuštění testu}

Nejpodstatnější operací je spuštění jednotlivých testů. O toto se stará metoda \inlinecode{LibState RunTest<T>(T testEnum, int timeout)}. Metoda očekává dva parametry:

\begin{itemize}
    \item \inlinecode{T testEnum} -- Identifikátor testu s generickým typem T, který typem musí být enumerátor.
    \item \inlinecode{int timeout} -- Maximální délka, po kterou služba očekává odpověď od účastníků testu. 
\end{itemize}

Metoda odesílá všem účastníkům testování direktivu k započnutí jednotlivých fází testu. Následně poté čeká na odpověď od všech účastníků testu. Doba čekání je určena argumentem \inlinecode{timeout}. Tento čas je vázaný na jednotlivá stádia testování. Tedy pokud metoda má časový limit 30 sekund, tak poté bude po každém stádiu testování čekat 30 sekund na odpověď od účastníků testu.

Metoda během běhu vyhodnocuje, zda se účastníka testování nedostal do chybové stavu. Na základě toho poté spouští jednotlivé fáze. Po skončení testu metoda vrací jednotu z hodnot enumerátoru \inlinecode{LibState}. 

\subsubsection{Ukončení testování}

Po dokončení celého testovacího běhu je zavolána metoda \inlinecode{Stop()}. Tato metoda odešle všem stále připojeným účastníkům direktivu k ukončení testování a ukončí spojení. Služba se snaží v případě chyby ukončit co nejvíce zařízení skrz stanovený protokol.

\subsection{Propojení s ostatními komponentami}

O propojení třídy \inlinecode{TestService} s ostatními komponentami knihovny se stará třída \inlinecode{ServiceRunner}.
Tato třída ve svém konstruktoru inicializuje konfiguraci služby a zjistí, zda je validní. Tento konstruktor přijímá jeden argument -- časový limit na inicializaci. V případě nezadání tohoto parametru třída využije defaultní časový limit definovaný ve třídě na 60 sekund. Následně konstruktor vytvoří instanci třídy \inlinecode{TestService} a zavolá metodu \inlinecode{Init}. V případě jakékoliv chyby je služba vyhodnocena jako v chybovém stavu. Toto určuje enumerátor \inlinecode{LibState} v proměnné \inlinecode{State}. Třída využívá pouze hodnoty \inlinecode{STATE\_OK} a \inlinecode{STATE\_ERROR}. Následný důvod vyhodnocení chyby je uloženo v proměnné \inlinecode{Response}. Třída následně má další metody, skrz které umožňuje ovládání instance třídy \inlinecode{TestService}. Tyto metody jsou:

\begin{itemize}
    \item \inlinecode{bool RunTest<T>(T test, int timeout = DefaultTimeout)} \\ Metoda předá instrukci pro spuštění testu. Očekává stejné parametry jako metoda \inlinecode{RunTest} třídy \inlinecode{TestService}. Jedinou změnou je, že pokud časový limit nebude zadán, tak bude použit defaultní časový limit.
    \item \inlinecode{bool AddVirtualParticipant(VirtualClient participant)} \\ Metoda spouští virtuální zařízení předané v argumentu a předává informaci o očekávání připojení virtuálního zařízení instanci třídy \inlinecode{TestService}. Instanci virtuálního zařízení následně uloží.
    \item \inlinecode{void TestCleanup()} \\ Metoda je volána po dokončení jednotlivého testu. Metoda předá instanci třídy \inlinecode{TestService} direktivu k odpojení virtuálních zařízení. Následně zkontroluje, zda se zařízení ukončila a pokud ne, tak zařízení ukončí.
    \item \inlinecode{void Stop()} \\ Metoda předá informaci o ukončení testovacího běhu.
\end{itemize}


\section{Rozhraní pro testovaná zařízení}

Rozhraní pro testovaná zařízení je vytvořeno dle definovaného návrhu. Rozhraní pro zařízení je implementováno pro jazyk \csharp{} a \cpp{}. Na výpisu \ref{listing:dev_if} můžeme vidět rozhraní v jazyce \cpp{}. 

\begin{listing}[htbp]
    \begin{minted}[breaklines]{cpp}
    class ITestClient
    {
    public:

        virtual ~ITestClient() {}

        virtual bool createConnection(const char* ipAddress, uint16_t port) = 0;

        virtual bool sendMessage(const Message& msg) = 0;

        virtual bool rcvMessage(Message& msg) = 0;

        virtual ITestCase* getTest(uint32_t test) = 0;

        virtual bool getMacAddress(uint8_t** macAddr, uint32_t* size) = 0;
    };
    \end{minted}
\caption{Ukázka definice rozhraní}
\label{listing:dev_if}
\end{listing}

Toto rozhraní je následně využito ve třídě \inlinecode{TestRunner}. Tato třída zajištuje běh jednotlivých účastníků testování. Třída má tyto metody:

\begin{itemize}
    \item \inlinecode{Start()} \\ Metoda vytvoří připojení s testovací službou.
    \item \inlinecode{HandleInstruction()} \\ Metoda přijímá instrukce od testovací služby a na základě nich provádí úkony.
    \item \inlinecode{RunTest(uint32\_t test\_enum, TestStateE test\_state)} \\ Metoda spouští jednotlivé části testů. 
    \item \inlinecode{Stop()} \\ Metoda ukončuje testovací běh.
\end{itemize}

Bližší vysvětlení si zaslouží metoda \inlinecode{RunTest}. Tato metoda je volána z metody \inlinecode{HandleInstruction}. K udržení stádia testu si třída mezi jednotlivými stádii drží instanci jednotlivých testů. Ve fázi přípravy na testování metoda zkontroluje že instance testu je nastavena na hodnotu \inlinecode{null}. Jiná hodnota by značila chybu během předchozího testování. Ve fázi úklidu po dokončení testu je odkaz na instanci testu nastaven na zpátky hodnotu null. 

\section{Virtuální zařízení}
Implementace virtuálního zařízení přímo kopíruje navrhnuté rozhraní pro testované zařízení. Logika virtuálního zařízení je obsažena ve třídě \inlinecode{VirtualClient}. Tato třída má tři možné konstruktory:

\begin{itemize} 
    \item \inlinecode{VirtualClient(ITestClient device)} \\
    Konstruktor dostane v argumentu odkaz na instanci rozhraní zařízení.
    \item \inlinecode{VirtualClient (ITestCase testCase)} \\
    Konstruktor dostane v argumentu odkaz na instanci jednoho testu, který je odvozený z rozhraní pro test.
    \item \inlinecode{VirtualClient (Func<UInt32, ITestCase> getTest)} \\
    Argumentem konstruktoru je funkce, která obsahuje seznam testů. Funkce v argumentu dostane číselnou reprezentaci testu a vrací instanci testu, který je odvozený z rozhraní testu.
\end{itemize}

Třída následně vytvoří instanci implementovaného rozhraní pro zařízení, které přijme v konstruktoru vybraný způsob reprezentace testů. Následně toto rozhraní předá instanci třídy \inlinecode{TestRunner} a po zavolání metody \inlinecode{Start} spustí \inlinecode{TestRunner} na vlastním vlákně.

Třída také obsahuje metodu \inlinecode{Stop}, která po zavolání kontroluje ukončení virtuálního zařízení v daném časovém limitu. Po vypršení časového limitu funkce ukončuje vlákno, na kterém instance třídy \inlinecode{TestRunner} běží.


\section{Propojení s frameworkem MSTest \note{Tahle část potřebuje předělat}}

Podstatnou součástí implementace je propojení služby s testovacím frameworkem MSTest. O toto se stará třída \inlinecode{API}. Jako jediná třída využívá nástroje tohoto frameworku. Třída je koncipovaná jako statická třída, i když je závislá na jím vytvořené instanci třídy \inlinecode{ServiceRunner}. Toto je uděláno z důvodu designového omezení frameworku MSTest. 


\subsection{Použití frameworku}
Framework MSTest využívá pro rozlišení testovacích součástí tzv. atributy. Atributy přidávají metadata k třídám, typům, metodám atd. \cite{attribute_docs}. Framework MSTest využívá například tyto atributy:

\begin{itemize}
    \item \inlinecode{AssemblyInitialize} -- identifikuje inicializační funkci, která je spuštěna před spuštěním testů \cite{attr_init_clean}
    \item \inlinecode{AssemblyCleanup} -- identifikuje funkci která bude spuštěna po skončení všech testů \cite{attr_init_clean}
    \item \inlinecode{TestClass} -- identifikuje třídu, která obsahuje testy \cite{mstest_docs}
    \item \inlinecode{TestMethod} -- identifikuje metody, které reprezentují jednotlivé  testy \cite{mstest_docs}
\end{itemize}

Framework MSTest následně používá tyto metody, které jsou identifikovány těmito a dalšími atributy. Při kompilaci jazyk \inlinecode{C\#} vytváří jednotlivé celky kompilovaného kódu, které poté spojuje do logického celku. Tyto jednotlivé celky se nazývají \textit{assembly}. \cite{assembly}

Testovací knihovna a použití frameworku MSTest tvoří samostatné celky -- \textit{assemblies}. MSTest framework používá správně definované funkce pouze ze své assembly. 


\subsection{Inicializace služby}

Při spuštění jednotlivých testů je nutné brát v potaz, že testovací služba neví jaké testy budou kdy spuštěny. Testovací služba se tedy musí používat nezávisle na všech testech. K tomuto využijeme dvě funkce frameworku MSTest -- \inlinecode{AssemblyInitialize} a \inlinecode{AssemblyCleanup}. Tyto dvě funkce jsou spuštěny ještě před započnutím a po skončení testovaní. V třídě \inlinecode{API} jejím implementacím odpovídají metody \inlinecode{AssemblyInit} a \inlinecode{AssemblyCleanup}. Metoda \inlinecode{AssemblyInit} přijímá dva argumenty:

\begin{itemize}
    \item \inlinecode{Assembly assembly} -- odkaz na runtime blok, který reprezentuje spuštěný blok programu mimo knihovnu
    \item \inlinecode{int InitTimeout} -- časový limit na inicializační fázi služby (tedy časový limit, který je využíván v metodě \inlinecode{TestService.Init}), pokud je ponechán prázdný, je použit defaultní časový limit z třídy \inlinecode{ServiceRunner}.
\end{itemize}

Metoda \inlinecode{AssemblyCleanup} symetricky ukončuje běh testovací služby. V určitých případech tato funkce nemusí být zavolána \question{Je zde potřeba citace}. Při zjištění chyby v běhu, která vyústí v ukončení testovacího běhu, je tato funkce volána manuálně.

Aby mohl MSTest rozeznat tyto funkce, je potřeba je definovat v assembly, která spouští tento framework. Toto vytváří problém u automatizace vytvoření propojení s testovací knihovnou. Pokud v testovací knihovně definujeme metody, které budou obsahovat atributy \inlinecode{AssemblyInitialize} a \inlinecode{AssemblyCleanup}, tak testovací framework  MSTest tyto metody nevidí a tím pádem je nepoužije. Proto je potřeba vytvořit dvě funkce, které budou obalovat metody \inlinecode{AssemblyInit} a \inlinecode{AssemblyCleanup}. Toto zpravuje třída \inlinecode{TestLibInit}. Tato statická třída pouze zaobaluje tyto dvě metody svými metodami a přidává k nim potřebné atributy. Tato třída není součásti assembly testovací knihovny. V kontextu knihovny je to pouze textový soubor. Připojení tohoto souboru k assembly, který spouští MSTest framework, bude přiblíženo v sekci o distribuci knihovny.

\subsection{Registrování a spuštění testů}

Při rozlišování jednotlivých testů knihovna a testovací služba využívá enumerátory. Tyto enumerátory odkazují na nějakou číselnou hodnotu, které je následně odesílána testovaným zařízením. Zároveň avšak může být využito více výčtových typů. Toto může zapříčinit vznik kolizí. 

K podchycení těchto kolizí je tedy potřeba udělat kontrolu před započnutím testování. Aby testovací služba mohla podchytit kolize, tak je potřeba rozlišit, které enumerátory identifikují testy. Proto testovací knihovna rozšiřuje sadu atributů o atribut \inlinecode{TestEnum}. Tento atribut říká, že enumerátor s atributem \inlinecode{TestEnum} reprezentuje nějaké testy. Tento atribut je povinný pro všechny enumerátory, které reprezentují testy.

Tato kontrola je provedena v inicializační části služby, uvnitř metody \inlinecode{AssemblyInit}. Díky předání runtime odkazu na assembly je možné zjistit, jaké enumerátory obsahují atribut \inlinecode{TestEnum}. Metoda následně zjistí, zda existují nějaké kolize mezi enumerátory s atributem \inlinecode{TestEnum} a pokud ano, tak vyhodí výjimku. Tato kontrola může být přeskočena předáním hodnoty \inlinecode{null} místo odkazu na runtime blok. 

Pro spuštění testování je použita metoda \inlinecode{Run} z třídy \inlinecode{API}. Tato metoda přijímá stejné argumenty jako metoda \inlinecode{RunTest} třídy \inlinecode{ServiceRunner}. Metoda zkontroluje, že argument identifikující test je typem enumerátor a obsahuje atribut \inlinecode{TestEnum}. V případě že se testovací služba nenachází v chybném stavu, tak metoda předá testovací službě direktivu k započnutí testu. Následně vyhodnotí úspěšnost testu. V případě, že služba je v chybném stavu, metoda vyhodnotí test jako bezvýsledný.

\todo{Doplnit}