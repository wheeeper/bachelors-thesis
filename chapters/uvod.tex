\begin{introduction}
Po industriální revoluci, která uvolnila dělníky z těžké manuální práce, je využití automatizace ve výrobě dalším velkým krokem ve vývoji průmyslu. 
Mezi výhody automatizace patří zefektivnění výroby, zrychlení výroby nebo snížení nákladů. Podstatným faktorem k ovládání zařízení, které se starají o automatizaci výroby, je spolehlivá komunikace. Toto zapříčinilo vznik průmyslových komunikačních sítí, z důvodu nedostatečnosti tehdy dostupných možností komunikace.
Tyto sítě --- v angličtině je jedna síť nazývána tzv. \textit{fieldbus} --- byli s postupem času standardizovány.

Při vývoji těchto zařízení, které se podílí na automatizaci výroby, je stejně důležité jako například jejich návrh také jejich testování. Hlavním úkolem této fáze je odhalení nedostatků produktu, které se liší od dané specifikace produktu. Testováním je tedy produkt kontrolován a z výsledků testů lze odvodit stav a kvalitu produktu. V dnešních době se u testování snaží využít výhod automatizace testování. Automatizace testů umožňuje jednoduchou opakovatelnost pravidelného testovaní, standardizuje jednotlivé testy a snižuje možnost vzniku lidské chyby při manuálním testování. V určitých případech zároveň umožnuje testovat případy, které nelze manuálně testovat.

Tato práce si stanovuje jako cíl vytvoření knihovny na automatizaci testů, které verifikují průmyslovou komunikaci. Hlavní motivací k vytvoření této knihovny je standardizace testování, zjednodušení, zrychlení vytváření testů a tím ušetření času testera. Tato knihovna je vytvářena pro společnost Siemens, s.r.o.

\todo{Dodat co obsahují jednotlivé kapitoly}
\end{introduction}
