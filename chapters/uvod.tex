\begin{introduction}

Jedním z cílů průmyslu je již od jeho vzniku zefektivnění výroby, které poté vede ke zvýšení zisků. V průběhu historie průmyslu se každá velká změna označuje za tzv. revoluci. V dnešní době se nacházíme ve čtvrté revoluci, která je často označována jako Průmysl 4.0. 

Cílem této revoluce je ještě větší automatizace opakujících se činností, kterou vykonávají lidé, digitalizace a zefektivnění komunikace mezi všemi zařízeními. Toto je ještě více podstatné v kontextu dnešní doby, kdy ve světě řádí pandemie, a výrobci si ještě více uvědomují křehkost lidské pracovní síly. Toto vše klade ještě větší požadavky na komunikaci v průmyslových sítích, ve kterých jsou používané specializované průmyslové protokoly. 

Při vývoji zařízení, které se podílí na automatizaci výroby, je stejně důležité jako například jejich návrh také jejich testování. Hlavním úkolem této fáze je odhalení nedostatků produktu, které se liší od dané specifikace produktu. Testováním je tedy produkt kontrolován a z výsledků testů lze odvodit stav a kvalitu produktu. V dnešních době se u testování snaží využít výhod automatizace testování. Mezi tyto výhody patří například jednoduchá opakovatelnost testování nebo umožnění častějšího testování. 

Tato práce se věnuje návrhu a implementaci knihovny, která bude automatizovat testy verifikace průmyslové komunikace. Hlavní motivací k vytvoření této knihovny je standardizace testování, zjednodušení a zrychlení vytváření testů. Toto poté vede k zjednodušení testování a snížení nákladů na testování. Tato knihovna je vytvářena pro společnost Siemens,~s.\,{}r.\,{}o.

Práce začíná kapitolou \ref{chap:cil}, ve které stanovuje cíle práce. V kapitole \ref{chap:teorie} práce definuje jednotlivé pojmy používané v práci a přibližuje kontext této práce. Následně kapitola \ref{chap:design} se věnuje návrhu testovací knihovny a všech jejich komponent. Následná implementace všech komponent je popsána v kapitole \ref{chap:implementation}. Použití vytvořeného řešení je následně popsáno v kapitole \ref{chap:demonstration}. V neposlední řadě kapitola \ref{chap:evaluation} se věnuje zhodnocení vytvořeného řešení a jeho přínosu.
\end{introduction}