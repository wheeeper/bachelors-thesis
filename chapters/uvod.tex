\begin{introduction}
Testování softwaru (v případě této práce primárně firmwaru) je velmi důležitá součást vývoje softwaru,
stejně jako například její samotný návrh a implementace. V této fázi vývoje je primárním cílem ověřit
funkcionalitu řešení a objevit chyby (tzv. bugy), které většinou vzniknou na základě lidské chyby a mohou
tím změnit požadované vlastnosti vyvíjeného softwaru. Automatizace testů poté umožňuje opakovatelnost 
pravidelného testovaní, standardizuje jednotlivé testy a snižuje možnost vzniku lidské chyby při 
manuálním testování. Některé aspekty softwaru ani bez automatizace není možné testovat. Tato práce se 
věnuje vývoji knihovny, která umožní automaticky testovat a verifikovat komunikaci na základě 
průmyslových protokolů. Tato testovací knihovna je vyvíjena pro firmu Siemens, s.r.o. \todo{Dodat co 
obsahují jednotlivé kapitoly}

\section{Cíl práce}
Cílem této práce je navrhnout a implementovat knihovnu, která umožní automatizovat testy verifikace průmyslové komunikace. 
Součástí této knihovny má být:
\begin{itemize}
    \item služba, která bude řídit testovací běh,
    \item rozhraní, které umožní implementaci knihovny na testovaném zařízení,
    \item protokol, který bude definovat komunikaci mezi službou a testovanými zařízeními.
\end{itemize}
Vytvořená knihovna poté má být propojena s Azure DevOps serverem tak, aby Azure DevOps server mohl 
následně automaticky spouštět testy. Dalším cílem je provést výzkum dostupných open-source knihoven pro 
průmyslové protokoly ModbusTCP a Ethernet/IP a vybrat vhodné kandidáty na implementaci do této knihovny. 
Součástí této práce má též být sada ukázkových testů pro jeden z průmyslových protokolů, na kterých bude 
vidět demonstrace funkcionality vyvinutého řešení. V neposlední řadě je cílem zhodnotit výsledné řešení z 
pohledu projektového řízení.
\end{introduction}
