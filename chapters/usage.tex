\chapter{Demonstrace použití knihovny}

V této kapitole se budu věnovat demonstraci použití knihovny při testování. K tomu použiji jeden z průmyslových protokolů.

\section{Průmyslové protokoly}
Jak jsem již definoval v sekci \ref{sec:fieldbus}, protokoly EtherNet/IP a ModbusTCP mají jasně stanovenou komunikaci. Jejich implementace avšak ale není triviální. Proto k vytvoření propojení na základě těchto protokolů použiji již dostupné open-source knihovny.  

\subsection{Průzkum open-source knihoven}
K tomu abych mohl správně ohodnotit dostupné open-source knihovny je potřeba definovat kritéria, na základě kterých budu jednotlivé knihovny hodnotit. Mezi tyto kritéria bude patřit:

\begin{itemize}
    \item Datum poslední aktualizace
    \item Náročnost využití knihovny, tedy náročnost implementace do knihovny a náročnost následného použití knihovny, hodnocené na stupnici 1--10, kde 1 značí nejjednodušší a 10 nejtěžší 
    \item Hodnocení dokumentace na stupnici 0--10, kde 0 značí, že knihovna neobsahuje žádnou dokumentaci a 10 značí nejkvalitnější dokumentaci
\end{itemize}

Zároveň při výběru knihovny vzniká implicitní požadavek, implementace průmyslového protokolu byla napsána v jazyce \csharp{} a byla kompatibilní s knihovnou.

\subsubsection{ModbusTCP}

V tabulce \ref{tab:modbus} můžeme vidět seznam nejvhodnějších knihoven pro využití v knihovně. Dle jednotlivých hodnocení můžeme vidět že knihovny se pohybují okolo stejné náročnosti na využití v knihovně. Všechny knihovny jsou dostupné jako NuGet balíčky, což jejich implementaci velice usnadňuje. Nejlepší dokumentaci obsahuje knihovna FluentModbus.

Jako nejvhodnější knihovnu jsem vyhodnotil knihovnu EasyModbusTCP od Rossmann Engineering. Tato knihovna jako jediná je vytvořena firmou, proto předpokládám, že na ni jsou kladeny vyšší kvalitativní standardy, než u ostatních knihoven, které jsou buď dílem jednotlivců a nebo komunitním projektem. 

\begin{table}[H]\centering
    \resizebox{\textwidth}{!}{
        \begin{tabular}{|C{3cm}|C{3cm}|C{2.1cm}|C{2.5cm}|C{2.5cm}|C{4cm}|}\hline
            Název & Autor & Poslední aktualizace & Náročnost využití & Hodnocení dokumentace &  Dostupné na adrese \\\hline 
            EasyModbusTCP & Rossmann Engineering & 31.12.2020 & 1 & 5 & \url{easymodbustcp.net}\\\hline
            FluentModbus & Apollo3zehn & 13.04.2021 & 2 & 6 & \url{github.com/Apollo3zehn/FluentModbus} \\\hline 
            NModbus & Rich Quackenbush & 14.07.2020 & 3 & 4 & \url{github.com/NModbus/NModbus} \\\hline 
            Modbus & Andres Müller & 13.04.2021 & 2 & 1 & \url{github.com/AndreasAmMueller/Modbus} \\\hline
        \end{tabular}
    }
    \caption{Seznam dostupných knihoven pro protokol ModbusTCP}
    \label{tab:modbus}
\end{table}

\subsubsection{EtherNet/IP}



\subsection{Použití knihovny} 