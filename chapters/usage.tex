\chapter{Demonstrace použití knihovny}

V této kapitole se budu věnovat demonstraci použití knihovny při testování. K tomu použiji jeden z průmyslových protokolů.

\section{Využití průmyslových protokolů}
Jak jsem již definoval v sekci \ref{sec:fieldbus}, protokoly EtherNet/IP a ModbusTCP mají jasně stanovenou komunikaci. Jejich implementace avšak ale není triviální. Proto k vytvoření propojení na základě těchto protokolů použiji již dostupné open-source knihovny.  

\subsection{Průzkum open-source knihoven}
K tomu abych mohl správně ohodnotit dostupné open-source knihovny je potřeba definovat kritéria, které musí knihovna splňovat. Jednotlivé knihovny musí být kompatibilní s vytvořenou testovací knihovnou. Tedy, knihovny by měli být implementovány v jazyce \csharp{} a kompatibilní s .NET Framework 4.8. 

Vybrané knihovny budu hodnotit na základě několika faktorů. Mezi tyto faktory bude patřit:

\begin{itemize}
    \item Datum poslední aktualizace
    \item Náročnost využití knihovny, tedy náročnost implementace do knihovny a náročnost následného použití knihovny, hodnocené na stupnici 1--10, kde 1 značí nejjednodušší a 10 nejtěžší 
    \item Hodnocení dokumentace na stupnici 0--10, kde 0 značí, že knihovna neobsahuje žádnou dokumentaci a 10 značí nejkvalitnější dokumentaci
\end{itemize}

Následně na základě těchto kritérií a mém osobním názoru doporučím knihovnu vhodnou k použití s testovací knihovnou. 

\subsubsection{ModbusTCP}

V tabulce \ref{tab:modbus} můžeme vidět seznam nejvhodnějších knihoven pro využití v knihovně. Dle jednotlivých hodnocení můžeme vidět že knihovny se pohybují okolo stejné náročnosti na využití v knihovně. Všechny knihovny jsou dostupné jako NuGet balíčky, což jejich implementaci velice usnadňuje.

Nejpoužívanější knihovnou, dle počtu stažení NuGet balíčku, je knihovna NModbus. Knihovna má průměrnou dokumentaci, kde popisuje jednotlivé třídy. Tato knihovna je udržována komunitně, a již dvakrát byla opuštěna předchozími autory. V kontextu toho, že knihovna má nejstarší datum poslední aktualizace, zde vzniká otázka, zda knihovna bude v budoucnu udržována.

Stejně jako knihovna NModbus, knihovny FluentModbus a Modbus jsou komunitním projektem. Knihovna Modbus ale bohužel neobsahuje žádnou dokumentaci až na krátkou ukázku kódu. Proto má knihovna v hodnocení zvýšenou náročnost na využití kvůli chybějící dokumentaci. Oproti tomu knihovna FluentModbus obsahuje nejlepší dokumentaci ze všech zmíněných knihoven. Bohužel knihovna FluentModbus aktuálně nepodporuje všechny funkce protokolu ModbusTCP. 

Jako nejvhodnější knihovnu jsem tedy vyhodnotil knihovnu EasyModbusTCP. Tato knihovna jako jediná je vytvořena firmou Rossmann Engineering. Z tohoto důvodu předpokládám, že na knihovnu jsou vyvíjeny vyšší kvalitativní nároky, než na ostatní knihovny.  

\begin{table}[H]\centering
    \resizebox{\textwidth}{!}{
        \begin{tabular}{|C{3cm}|C{3cm}|C{2.1cm}|C{1.8cm}|C{2.3cm}|C{4cm}|}\hline
            Název & Autor & Poslední aktualizace & Náročnost využití & Hodnocení dokumentace &  Dostupné na adrese \\\hline 
            EasyModbusTCP & Rossmann Engineering & 31.12.2020 & 1 & 5 & \url{easymodbustcp.net}\\\hline
            FluentModbus & Apollo3zehn, et al. & 13.04.2021 & 1 & 6 & \url{github.com/Apollo3zehn/FluentModbus} \\\hline 
            NModbus & Rich Quackenbush, et al. & 14.07.2020 & 2 & 4 & \url{github.com/NModbus/NModbus} \\\hline 
            Modbus & Andres Müller, et al. & 13.04.2021 & 3 & 1 & \url{github.com/AndreasAmMueller/Modbus} \\\hline
        \end{tabular}
    }
    \caption{Seznam dostupných knihoven pro protokol ModbusTCP}
    \label{tab:modbus}
\end{table}

\subsubsection{EtherNet/IP}



\subsection{Použití knihovny} 