\chapter{Teoretická část} % Je to vhodný název kapitoly?

\section{Průmyslová komunikace}

Při řešení průmyslové komunikace se často objevuje slovo \textit{fieldbus}. Běžný význam tohoto slova je \uv{Síť, která propojuje průmyslová zařízení jako kontrolery, PLC, regulátory atd.} \cite{fieldbus_thomesse}. Vznik těchto sítí je úzce spojený s historií vývoje informačních technologií. V době, kdy začali tyto sítě vznikat, nebyli dostupné komunikační technologie, jako dnes. Dostupné informační a telekomunikační sítě té doby nemohli uspokojit potřeby průmyslových sítí na deterministickou, spolehlivou a efektivní komunikaci \cite{future_of_ind_com}. 

V dobách vzniku těchto komunikačních protokolů bylo pro fyzické připojení využíváno různých protokolů -- například RS-232 nebo RS-485. Nedostatkem těchto protokolů avšak byli vysoké implementační náklady. S růstem popularity standardu IEEE 802.3 -- dnes často nazýván Ethernet -- se začalo zkoumat možné použití tohoto standardu při průmyslové komunikaci. Odborníci došli k závěru, že je možné při průmyslové komunikace v reálném čase využít tento standard \cite{lee_ethernet_fieldbus}. V dnešní době je Ethernetový standard hojně využíván při přenosu průmyslové komunikace.

Hlavním cílem naší práce je vytvoření testovací knihovny pro ET 200SP I/O systém, vyvíjený společností Siemens,~s.\,{}r.\,{}o. Toto zařízení využívá tzv. \textit{Multifieldbus technologie}. Tato technologie umožňuje zařízení komunikovat na základě několika průmyslových protokolů. V rámci této práce budeme využívat protokoly ModbusTCP a Ethernet/IP. Oba tyto protokoly, jak již z jejich názvů plyne, využívají Ethernetový standard.  

% MiDu: udělal bych dvě krátké podkapitolky jednu na Ethernet / IP a druhou na Modbus TCP a v krátkosti bych je popsal -> taková stručná charakteristika... nesel bych do historie nepopisoval bych vznik ale technicky bych je popsal v krátkosti jak funguji co využívají a třeba nějaké charakteristiky... na netu se toho da vygooglit spousta

\subsection{ModbusTCP}
Modbus je komunikační protokol, který je umístěn na aplikační vrstvě ISO/OSI modelu. Tento protokol, vytvořený v roce 1979, umožňuje komunikaci klient-server mezi zařízeními na různých typech sítí a sběrnic. Modbus používá komunikaci požadavek-odpověď. Podporované služby jsou poté definováno funkčními kódy, které jsou součástí požadavku.

Protokol definuje strukturu zprávy (tzv. PDU -- Protocol Data Unit) nezávisle od komunikační vrstvy. Tato zpráva se avšak může rozšířit, v závislosti na způsobu přenosu této zprávy. V závislosti na typu sítě je poté tato zpráva rozšířena o další údaje. Tento celek se poté označuje jako ADU -- Application Data Unit. Tuto celou strukturu můžeme vidět znázorněnou na obrázku \ref{fig:modbus_frame}. \cite{modbus}

\begin{figure}[htbp]
    \centering 
    \includegraphics[width=\textwidth]{assets/img/modbusframe.pdf}
    \caption{Obecné znázornění jednoho rámce protokolu Modbus}
    \source{Vytvořeno dle předlohy z \cite{modbus}}
    \label{fig:modbus_frame}
\end{figure}

\todo{Doplnit}


\subsection{Ethernet/IP}
\todo{Doplnit}

\section{Testování}

Testování je podstatnou součástí vývoje produktu a jeho softwaru. Cílem testování není pouze odhalení chyb v softwaru, ale také verifikace a validace softwaru \cite{singh2012software}. Při testování se snažíme vytvářet situace, ve kterých můžeme ověřit, že se software chová dle zadané specifikace.

Testování softwaru je zároveň dovednost. Při testování musí tester vybrat z nekonečného množství možných testů nějaký konečný počet, který nejlépe reprezentuje danou problematiku a pokrývá co největší možnou množinu všech možných případů. Zároveň musí vzít v potaz náročnost na vytvoření testu a na rigidnost vytvořeného testu proti změnám v softwaru. Tyto faktory poté ovlivňují i náklady na testování. \cite{fewster1999software}

Od testování softwaru se také odvodit kvalita softwaru. Kvalitu softwaru se dá určit tím, jak moc vytvořený software odpovídá zadaným specifikacím \cite{software_quality}. Tyto informace jsou poté velmi důležité pro managment. Díky nim může vyhodnocovat současný stav vývoje a upravovat plán na vývoj. 

Testování zároveň zvyšuje důvěru ve vyvíjený softwaru. Každý dobře navržený test snižuje šanci, že v softwaru existuje nepodchycená chyba. S každým rozsáhlým testováním se tato důvěra zvyšuje. \cite{fewster1999software}



\subsection{Rozdělení testů}

I když primární cíl testování je jednotný, přístupů k testování je několik. Vhodnost jednotlivých přístupů se mění na základě testované komponenty. Tyto přístupy se dají podle \cite{luo2001software} rozdělit do několika kategorií.

\subsubsection{Podle znalosti komponenty}

Testování se dá rozdělit podle přístupu k informacím, které o komponentách softwaru/systému víme. Tyto typy jsou:

\begin{description}
    \item[Black box testování] Nazýváno taktéž funkční testování. Na software se pohlíží jako na tzv. černou skříňku. O komponentě nebo celku nic nevíme a testujeme na základě funkcionálních požadavků, návrhu a specifikací. 
    \item[White box testování] Se znalostí implementace testované části se snažíme vytvořit takové testy, které způsobí spouštění určitých částí testované komponenty. Cílem je co největší pokrytí testování dané komponenty.
    \item[Grey box testování] Kombinace Black box a White box testování. Při testování máme nějakou znalost implementace komponenty, ale je nižší, než při White box testování. \cite{khan2010different}
\end{description}

\subsubsection{Podle částí vývoje}

Testování podle částí vývoje se přibližuje vývojovému cyklu. Tyto kategorie jsou:  

\begin{description}
    \item[Testování jednotlivých částí] V angličtině nazýváno známým pojmem \uv{Unit testing}. Je to nejnižší úroveň testování. Testuje jednotlivé komponenty systému samostatně.
    \item[Integrační testování] Testování dvou a více komponent, které spolu vytváří nějaký větší celek softwaru. Často také využíván při testování částí, které nelze samostatně testovat.
    \item[Systémové testování] Testování softwaru jako celku. Testování se směřuje na testování funkčních požadavků. Zároveň je možno vyhodnocovat další požadavky na systém, jako spolehlivost, bezpečnost, atd.
    \item[Akceptační testování] U toho testování se systém dostane do rukou zákazníkovi/uživatelům. Cílem je otestování produktu u uživatelů softwaru a získání jejich zpětné vazby. 
\end{description}

Propojení testů a vývojového cyklu je dobře znázorněno na tzv. V-modelu, který můžeme vidět na obrázku \ref{fig:vmodel}. Na tomto modelu, nazývaném podle svého tvaru, můžeme vidět jednotlivé typy stádia vývoje a jejich korespondující ověřování v závislosti na čase.

\begin{figure}[htbp]
    \centering 
    \includegraphics[width=0.9\textwidth]{assets/img/vmodel.pdf}
    \caption{V-model}
    \source{Doplnit}
    \label{fig:vmodel}
\end{figure}


\section{Automatizace testování}

Automatizace testování je odlišná od samotného testování. Automatizace testu neurčuje samotnou kvalitu testu. \highlight{Pokud automatizujeme test, který nic nového nepřinese, dostaneme toto nic pouze rychleji} \cite{fewster1999software}. Automatizování je ale přesto v mnoha ohledech v dnešní době standardem při testování, a to především díky jeho výhodám. Mezi tyto výhody podle \cite{fewster1999software} patří:

\begin{description}
    \item[Častější testování] S automatizací jsme schopni testovat software mnohem častěji, než při manuálním testování. Software může být testován například při každé jeho změně. Toto manuálně je velmi náročné, už jenom kvůli vysokému požadavku na lidské zdroje.  
    \item[Možnost testovat \highlight{nové} věci] Automatizace umožňuje testovat takové aspekty, které se nedají manuálně testovat, nebo jejich manuální testování je velice náročné. Tím se zvyšuje pokrytí testování.
    \item[Lepší využití zdrojů] Tester je vysoce kvalifikovaný člověk a jeho využití na opakované vkládání vstupů a ověřování výstupů je v některých případech plýtvání jeho drahocenným časem. Díky automatizaci se tester může zaměřit na jiné přínosnější činnosti, jako například vytváření nových testů, kterými pokryje nové případy.
    \item[Konzistence] Při automatizaci testování každý běh testu proběhne naprosto identicky. Stejně jako ve vývoji, i v testování může dojít k lidské chybě. Díky automatizaci se šance lidské chyby snižuje. Toto zvyšuje konzistenci testování, než když se testy provádějí manuálně.
    \item[Snížení doby testování] Jednou automatizované testy můžou být provedeny mnohem rychleji a efektivněji, než při jejich manuálním spuštění. Toto způsobuje snížení potřebné doby na testování.
\end{description}

\section{Azure DevOps}

Azure DevOps server poskytuje vývojářům služby, které pomáhají při vývoji softwaru. Jeho cílem je podporovat jednotlivé procesy vývoje, což poté zrychluje vývoj softwaru. \cite{azure_devops} V této práci budu využívat několik služeb, které Azure DevOps server nabízí. Tyto služby budou:

\begin{description}
    \item[Azure Repos] Azure Repos je sada nástrojů, která umožňuje správu jednotlivých verzí softwaru. V této práci budeme používat systém správy verzí Git, který je Azure Repos podporovaný. \cite{azure_repos}
    \item[Azure Pipelines] Azure Pipelines automaticky kompiluje a testuje vyvíjený software. Zároveň podporuje procesy jako kontinuální integrace, kontinuálního nasazení softwaru nebo kontinuální testování. \cite{azure_pipelines}
    \\ \question{Je potřeba vysvětlit co je CI a CD, když to nikde jinde nepoužívám}
    \item[Azure Test Plans] Azure Test Plans přináší sadu nástrojů, které umožnují spravovat testování softwaru. Tyto nástroje umožňují správu testovacích sad a jednotlivých testů, ať už manuálních nebo automatizovaných. \cite{azure_test_plans}
    \item[Azure Artifacts] Azure Artifacts umožňuje publikování a verzování různých typů balíčků. Následně tato správa těchto balíčků může být využita při vydání těchto balíčků. Mezi podporované balíčky patří například NuGet, Maven nebo npm. Zároveň skrz Azure Artifact můžeme publikovat data z Azure Pipelines. \cite{azure_artifacts}
\end{description}

Azure DevOps server využívá při delegování úkolů tzv. agenty. Agent je výpočetní infrastruktura s nainstalovaným softwarem agenta, který pracuje na jedné určité úloze \cite{agent_docs}. Agent následně provádí například jednotlivé úkony definované v Azure Pipelines, jako kompilace, testování atd.


\section{Framework MSTest}

Framework MSTest je výchozí testovací framework, který je integrován do IDE Visual Studio. Díky tomu je také často nazýván jako \uv{Visual Studio Unit Testing Framework}. Framework započal jako nástroj spouštěný z příkazové řádky, který následně prováděl testování. Díky implementaci do Visual Studia je tento framework často preferován vývojáři, kteří používají Visual Studio pro vývoj. 

MSTest framework přináší nástroje, které jsou potřeba k verifikaci a validaci softwaru. V dnešní době framework MSTest V2 je open-source projekt, který je stále rozvíjen. Mezi jeho výhody patří podpora napříč platformami a rozšiřitelnost. \cite{mstest_descr}