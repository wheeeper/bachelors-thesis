\chapter{Úvod do problematiky} % Je to vhodný název kapitoly?


Po industriální revoluci, která uvolnila dělníky z těžké manuální práce, je využití automatizace ve výrobě dalším velkým krokem ve vývoji průmyslu \cite{roots_of_automation}. 
Mezi výhody automatizace patří zefektivnění výroby, zrychlení výroby nebo snížení nákladů. Podstatným faktorem k ovládání zařízení, které se starají o automatizaci výroby, je spolehlivá komunikace. Toto zapříčinilo vznik průmyslových komunikačních sítí, z důvodu nedostatečnosti tehdy dostupných možností komunikace \cite{evolution_of_factory_automation}.
Tyto sítě --- v angličtině je jedna síť nazývána tzv. \textit{fieldbus} --- kladou důraz na spolehlivost, efektivnost a determinismus komunikace \cite{future_of_ind_com}. Kontrolování komunikace a její spolehlivosti při vývoji je tedy o to víc podstatné, než například u osobních systémů. Možná nepodchycená chyba totiž může v reálném provozu způsobit velké škody, ať už materiální nebo peněžité.


\section{Testování}

% Warning: Trochu sekundární citace {test_automation}

Testování softwaru je jednou z velmi důležitých součástí vývoje softwaru. Podílí se na 30--60 \% z celkových nákladů vývoje, v závislosti na komplexitě a kritičnosti produktu.
Hlavním úkolem testování je odhalit chyby, které mohou vzniknout během vývoje softwaru, a to co nejdříve od vzniku chyby \cite{test_automation}. Od testování softwaru se také odvodit kvalita softwaru. Kvalitu softwaru se dá určit tím, jak moc vytvořený software odpovídá zadaným specifikacím \cite{software_quality}. Testování softwaru je tedy přímo propojeno se softwarovou kvalitou a ve fázi testování měříme, jak moc se software blíží k jeho specifikacím \cite{KUMAR20168}. Vytváření jednotlivých testů je též balancování několika faktorů:
\begin{itemize}
    \item jak nákladné je vytvoření testů,
    \item jak náročné je analyzování výsledku testu,
    \item jak je test rigidní na průběžné změny vyvíjeného softwaru \cite{fewster1999software}.
\end{itemize}


\subsection{Automatické testování}


