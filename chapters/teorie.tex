\chapter{Teoretická část} % Je to vhodný název kapitoly?

\dummytext{1}

\section{Testování}

% Warning: Trochu sekundární citace {test_automation}

Testování softwaru je jednou z velmi důležitých součástí vývoje softwaru. Podílí se na 30--60 \% z celkových nákladů vývoje, v závislosti na komplexitě a kritičnosti produktu.
Hlavním úkolem testování je odhalit chyby, které mohou vzniknout během vývoje softwaru, a to co nejdříve od vzniku chyby \cite{test_automation}. Od testování softwaru se také odvodit kvalita softwaru. Kvalitu softwaru se dá určit tím, jak moc vytvořený software odpovídá zadaným specifikacím \cite{software_quality}. Testování softwaru je tedy přímo propojeno se softwarovou kvalitou a ve fázi testování měříme, jak moc se software blíží k jeho specifikacím \cite{KUMAR20168}. Vytváření jednotlivých testů je též balancování několika faktorů:
\begin{itemize}
    \item jak nákladné je vytvoření testů,
    \item jak náročné je analyzování výsledku testu,
    \item jak je test rigidní na průběžné změny vyvíjeného softwaru \cite{fewster1999software}.
\end{itemize}


\section{Automatizace testování}


