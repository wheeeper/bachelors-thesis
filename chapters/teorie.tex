\chapter{Teoretická část} % Je to vhodný název kapitoly?

\section{Testování}

Testování je podstatnou součástí vývoje softwaru. Cílem testování není pouze odhalení chyb v softwaru, ale také verifikace a validace softwaru \cite{singh2012software}. Při testování se snažíme vytvářet situace, ve kterých můžeme ověřit, že se software chová dle zadané specifikace.

Testování softwaru je zároveň dovednost \cite{fewster1999software}. Při testování člověk musí vybrat z nekonečného množství možných testů nějaký konečný počet, který nejlépe reprezentuje danou problematiku a pokrývá co největší možnou množinu všech možných případů. Zároveň musí vzít v potaz náročnost na vytvoření testu a na rigidnost vytvořeného testu proti změnám v softwaru. Tyto faktory poté ovlivňují i náklady na testování. Od testování softwaru se také odvodit kvalita softwaru. Kvalitu softwaru se dá určit tím, jak moc vytvořený software odpovídá zadaným specifikacím \cite{software_quality}. Tyto informace jsou poté velmi důležité pro managment a pro další plánování vývoje. 

I když cíl testování je jednotný, přístupů k testování je několik. Vhodnost jednotlivých přístupů se mění na základě testované komponenty. Tyto přístupy se dají rozdělit do několika kategorií \cite{luo2001software}.

\subsection{Podle znalosti komponenty}

Testování se dá rozdělit podle přístupu k informacím, které o komponentách softwaru/systému víme. Tyto typy jsou:

\begin{description}
    \item[Black box testování] Nazýváno taktéž funkční testování. Na software se pohlíží jako na tzv. černou skříňku. O komponentě nebo celku nic nevíme a testujeme na základě funkcionálních požadavků a návrhu. 
    \item[White box testování] Se znalostí implementace testované části se snažíme vytvořit takové testy, které způsobí spouštění určitých částí testované komponenty. Cílem je co největší pokrytí testování dané komponenty.
    \item[Grey box testování] Kombinace Black box a White box testování. Při testování máme nějakou znalost implementace komponenty, ale je nižší, než při White box testování \cite{khan2010different}.
\end{description}

\subsection{Podle částí vývoje}

Testování podle částí vývoje se přibližuje vývojovému cyklu. Tyto kategorie jsou:  

\begin{description}
    \item[Testování částí] V angličtině nazýváno známým pojmem \uv{Unit testing}. Je to nejnižší úroveň testování. Testuje jednotlivé komponenty systému samostatně.
    \item[Integrační testování] Testování dvou a více komponent, které spolu vytváří nějaký větší celek softwaru. Často také využíván při testování částí, které nelze samostatně testovat.
    \item[Systémové testování] Testování softwaru jako celku. Testování se směřuje na testování funkčních požadavků. Zároveň je možno vyhodnocovat další požadavky na systém, jako spolehlivost, bezpečnost, atd.
    \item[Akceptační testování] U toho testování se systém dostane do rukou zákazníkovi/uživatelům. Cílem je otestování produktu u potencionálních uživatelů softwaru a získání jejich zpětné vazby. 
\end{description}

\subsection{Analýza softwaru}
Součástí testování je i analýza softwaru. Tato analýza se dá rozdělit podle toho, zda je potřeba daný vyvíjený software vůbec spouštět. Tyto kategorie jsou:

\begin{description}
    \item[Statická analýza] Tato analýza je prováděno bez spuštění softwaru. Analýza je prováděna na napsaný kód. Vyhodnocovány jsou obecné vlastnosti napsaného kódu bez znalosti kontextu použití.
    \item[Dynamická analýza] Testování je provedeno spuštěním softwaru a použitím reálných metod systému s reálnými daty v simulovaných situacích. 
\end{description}

\section{Automatizace testování}

Automatizace testování je odlišná od samotného testování. Automatizace testu neurčuje samotnou kvalitu testu. Automatizováním testu, který nic nového nepřinese, dostaneme toto nic pouze rychleji \cite{fewster1999software}. Automatizování je ale přesto v mnoha ohledech v dnešní době standardem při testování, a to především díky jeho výhodám. Díky automatizaci jsme při vývoji schopni provádět opakované testy za frakci ceny, než kdyby byli prováděny manuálně. Toto zároveň uvolňuje testery ke směřování své snahy k rozšiřování množiny testů a tím pokrytí co nejvíce případů.

\todo{Rozšíření textu o automatizaci}


\section{Průmyslová komunikace}

Při řešení průmyslové komunikace se často objevuje slovo \textit{fieldbus}. Běžný význam tohoto slova je \uv{Síť, která propojuje průmyslová zařízení jako kontrolery, PLC, regulátory atd.} \cite{fieldbus_thomesse}. Vznik těchto sítí je úzce spojený s historií vývoje informačních technologií. V době, kdy začali tyto sítě vznikat, nebyli dostupné komunikační technologie, jako dnes. Dostupné informační a telekomunikační sítě té doby nemohli uspokojit potřeby průmyslových sítí na deterministickou, spolehlivou a efektivní komunikaci \cite{future_of_ind_com}. 

V dnešní době jsou tyto průmyslové sítě mezinárodně standardizovaný. Jako příklad protokolů můžeme uvést ModbusTCP, nebo Ethernet/IP, které už využívají výhod Ethernet připojení a zároveň satisfakují průmyslové potřeby. 
