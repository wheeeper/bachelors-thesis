\chapter{Teoretická část} % Je to vhodný název kapitoly?

\section{Testování}

Testování je podstatnou součástí vývoje softwaru. Cílem testování není pouze odhalení chyb v softwaru, ale také verifikace a validace softwaru \cite{singh2012software}. Při testování se snažíme vytvářet situace, ve kterých můžeme ověřit, že se software chová dle zadané specifikace.

Testování softwaru je zároveň dovednost \cite{fewster1999software}. Při testování člověk musí vybrat z nekonečného množství možných testů nějaký konečný počet, který nejlépe reprezentuje danou problematiku a pokrývá co největší možnou množinu všech možných případů. Zároveň musí vzít v potaz náročnost na vytvoření testu a na rigidnost vytvořeného testu proti změnám v softwaru. Tyto faktory poté ovlivňují i náklady na testování. Od testování softwaru se také odvodit kvalita softwaru. Kvalitu softwaru se dá určit tím, jak moc vytvořený software odpovídá zadaným specifikacím \cite{software_quality}. Tyto informace jsou velmi důležité pro managment a pro další plánování vývoje. 

\section{Automatizace testování}

Automatizace testování je odlišná od samotného testování. Automatizace testu neurčuje samotnou kvalitu testu. Automatizováním testu, který nic nového nepřinese, dostaneme toto nic pouze rychleji \cite{fewster1999software}. Automatizování je ale přesto v mnoha ohledech v dnešní době standardem při testování, a to především díky jeho výhodám. Díky automatizaci jsme při vývoji schopni provádět opakované testy za frakci ceny, než kdyby byli prováděny manuálně. Toto zároveň uvolňuje testery ke směřování své snahy k rozšiřování množiny testů a tím pokrytí co nejvíce případů.


\section{Průmyslová komunikace}

Při řešení průmyslové komunikace se často objevuje slovo \textit{fieldbus}. Běžný význam tohoto slova je \uv{Síť, která propojuje průmyslová zařízení jako kontrolery, PLC, regulátory atd.} \cite{fieldbus_thomesse}. Vznik těchto sítí je úzce spojený s historií vývoje informačních technologií. V době, kdy začali tyto sítě vznikat, nebyli dostupné komunikační technologie, jako dnes. Dostupné informační a telekomunikační sítě té doby nemohli uspokojit potřeby průmyslových sítí na deterministickou, spolehlivou a efektivní komunikaci \cite{future_of_ind_com}. 

V dnešní době jsou tyto průmyslové sítě mezinárodně standardizovaný. Jako příklad protokolů můžeme uvést ModbusTCP, nebo Ethernet/IP, které už využívají výhod Ethernet připojení a zároveň satisfakují průmyslové potřeby. 
