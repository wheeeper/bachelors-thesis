\begin{conclusion}
Cílem této práce bylo vytvořit testovací knihovnu, která umožní automatizovat testy verifikace průmyslové komunikace. Tato knihovna následně měla být propojena s Azure DevOps serverem tak, aby Azure DevOps server mohl automaticky testy spouštět a poskytovat uživateli informace o průběhu jednotlivých testů. Dalším cílem práce bylo prozkoumat dostupné open-source knihovny pro průmyslové protokoly ModbusTCP a EtherNet/IP a za pomoci jedné z knihoven demonstrovat funkcionalitu vytvořeného řešení. V neposlední řadě bylo cílem zhodnotit vytvořené řešení a jeho přínosy.

Cíle této práce byly splněny. Práce v kapitole \ref{chap:teorie} definuje jednotlivé pojmy a přibližuje kontext této práce. Testovací knihovna a všechny její části byly následně v kapitole \ref{chap:design} navrhnuty. Testovací služba, která řídí testovací běh, byla navrhnuta jako doplněk do testovacího frameworku MSTest. Následně za pomoci tohoto frameworku bylo řešení propojeno se serverem Azure DevOps, který mohl jednotlivé testy registrovat, automaticky spouštět a zobrazovat jejich výsledky. Implementace všech částí testovací knihovny byla následně popsána v kapitole \ref{chap:implementation} a funkčnost vytvořeného řešení demonstrována v kapitole \ref{chap:demonstration}. Kapitola \ref{chap:evaluation} následně zhodnotila vytvořené řešení, jeho přínosy a úskalí z pohledu projektového managementu. 

Vytvořené řešení splňuje požadavky, které na něj byly kladeny, avšak je zde stále prostor pro zlepšení. Zřejmým rozšířením knihovny je implementace jednotlivých testovacích partnerů, kteří budou simulovat chování jednotlivých zařízení. Zároveň je možné zlepšit registraci testů, respektive automatizovat tuto registraci, na testovaném zařízení. 
\end{conclusion}