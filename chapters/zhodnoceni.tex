\chapter{Vyhodnocení vytvořeného řešení}\label{chap:evaluation}

V této kapitole se věnuji zhodnocení vytvořeného řešení a jeho přínosu. \\


\section{Kontext}
Vytvořená testovací knihovna míří na automatizaci testů verifikace průmyslové komunikace. Tato automatizace je následně využito při procesech kontinuální integrace a kontinuálního testování. 

Tyto procesy přináší do vývoje softwaru několik výhod. Ze studie, která zkoumala dva porovnatelné projekty, kdy jeden z projektů používal kontinuální integraci a kontinuální testování oproti druhému projektu, víme, že za pomoci těchto procesů byl projekt s pomocí těchto procesů schopen odhalit mnohem více chyb před vydáním softwaru oproti druhému projektu. Zároveň projekt bez zmíněných procesů potřeboval na odstranění chyb více času oproti druhému projektu. \cite{ci_study}

Toto je ještě více podstatné v kontextu testovaného produktu. Na industriální zařízení jsou kladeny mnohem větší požadavky na stabilitu a spolehlivost. Odhalení chyby při ostrém provozu může vést k ekonomické škodě zákazníka, ať už to jsou v lepším případě ztráty zisku a nebo horším případě materiální škody.

Důležité je avšak také implementace těchto procesů. Logickým cílem každé firmy je implementovat tyto procesy za co nejméně peněz a námahy. Přesně na tyto aspekty míří vytvořená testovací knihovna. 


\section{Zhodnocení}

Automatizované testy verifikace průmyslové komunikace vyžadují určitý počet zařízení (v kontextu vytvořené knihovny účastníků), které následně komunikují s testovaným produktem. Počet a typ zařízení se následně může měnit v závislosti na jednotlivých testovaných případech.  

To ale vytváří vysoké hardwarové na nároky na testování. Tedy k tomu, aby bylo možné automatizovaně testovat, je potřeba vytvářet sestavy zařízení, na kterých je následně testováno. Mnohdy je pro firmu výhodnější vytvořit více sestav v různých konfiguracích než míti jednu sestavu a tu pravidelně přestavovat. Přestavováním sestav by zároveň testování už nebylo plně automatizované. 

Nákup zařízení, které se podílí na testovaní vyvíjeného projektu, je nemalou položkou v rozpočtu projektu. Možnou úsporu zařízení si můžeme ukázat na jedné z investic do projektu. Seznam nakoupených zařízení můžeme vidět na obrázku \ref{tab:device_list}. Zároveň jsou k ovládání těchto zařízení potřeba licence na software, který je může ovládat. Seznam licencí potřebných k ovládaní zařízení v tabulce \ref{tab:device_list} můžeme vidět v tabulce \ref{tab:device_licenses}. 

Za pomocí testovací knihovny prostřednictvím testovacího partnera je možné tyto zařízení simulovat. Simulací těchto zařízení můžeme náklady na testování výrazně snížit. Cílem knihovny ale není použití fyzických zařízení úplně odstranit. Existují případy, kdy tyto zařízení budou stále potřeba. 

Jak ale můžeme v tabulkách \ref{tab:device_list} a \ref{tab:device_licenses} vidět, spousta věcí je potřeba ve větším množství. Pokud za pomoci testovací knihovny snížíme počet zařízení tak, že odstraníme potřebu duplicitních zařízení, můžeme v rozpočtu projektu ušetřit až \koruna{252767}. Tím také snižujeme nároky na licence, jelikož méně zařízení vyžaduje méně lidí, které je současně konfigurují. Pokud budeme předpokládat, že počet potřebných licencí se sníží na polovinu, tak poté se v rozpočtu projektu může ušetřit až \koruna{130034,64}.

Je ale potřeba také vzít potaz další součásti, které jsou potřeba k zprovoznění těchto zařízení. Využitím méně zařízení můžeme zároveň ušetřit na ostatních materiálech, jako jsou ethernetové kabely atd. Zároveň je potřeba vzít v potaz čas, který je třeba k stavbě, modifikací a údržbě systémů a zařízení v nich. Toto se stává ještě obtížnější v době, kdy ve světě je pandemie a většina práce je prováděna z domova. Využitím testovací knihovny také místo hardwarových problémů vznikají problémy softwarové, které se dají lépe podchytit a efektivněji řešit.
Jednotlivá zařízení můžou zároveň selhat, což poté vyžaduje jejich opětovnou koupi.




% v aktuálním stavu vytváří vysoké hardwarové nároky. 


% V aktuálním řešení automatizovaného testování při vývoji testovaného produktu jsou potřeba zařízení, která komunikují s testovaným produktem. To vytváří hardwarové nároky na testování. Zároveň je potřeba tyto systémy sestavit a nakonfigurovat, což není triviální záležitost. 
% \todo{Zjistit od Jardy jak se to s přestavováním}

% S použitím testovací knihovny a virtualizace za pomoci testovacího partnera můžeme chování jednotlivých zařízení simulovat jejich chování a tím snížit hardwarové a ekonomické nároky na testování. Cílem knihovny není odstranění potřeby k hardwarovým zařízením, protože v určitých testovacích případech jsou stále potřeba.



% Jejím využitím se usnadňuje testování a za 


% Využití knihovny testování výrazně usnadňuje a to díky jednoduchosti použití knihovny. Implementace na testovaném zařízení je závislá pouze na TCP/IP protokolu a jím vytvářeném připojení. Zároveň je díky ostatním implementovaným komponentám tato integrace časově nenáročná.

% Stejně jednoduchá je implementace řídící testovací služby. Díky distribuci prostřednictvím NuGet balíčků je implementace ve finále pouze několik kliknutí. 

% S využitím testovací knihovny standardizujeme vytváření testů. Testy na každé zařízení jsou vytvářeny identicky, což může zrychlit vytváření těchto testů. I když je knihovna mířena na verifikaci průmyslové komunikace, jeho jádro může být využito k více aspektům testování, jako je například testování jednotlivých částí. 



% \begin{table}[htbp]
%     \centering
%     \resizebox{\textwidth}{!}{
%     \begin{tabular}{|C{3cm}|C{4cm}|C{3.5cm}|}\hline
%         \textbf{Výrobce} & \textbf{Model} & \textbf{Cena za kus bez DPH}\\\hline
%         Allen-Bradley & 1783-BMS20CGN & \koruna{64480.27}\\ \hline
%         Beckhoff & CX2030-0125 & \koruna{66650.44}\\\hline
%         Beckhoff & CX2100-0004 & \koruna{8663.21}\\\hline
%         Schneider & BMXP342020 340-20 & \koruna{30536.8}\\\hline
%         Schneider & BMXCPS2000 & \koruna{5220.27}\\\hline
%         Schneider & BMXXBP0400 & \koruna{3493.03}\\\hline
%         Schneider & M251MESC & \koruna{8192.46}\\\hline
%         Schneider & TM262L20MESE8T & \koruna{36636.86}\\\hline
%         Rockewell & 1769-L24ER-QB1B & \koruna{38526.54}\\\hline
%     \end{tabular}
%     }
%     \caption{Tabulka zařízení potřebných k testování bez množství}
%     \label{tab:device_prices}
% \end{table}

\begin{table}[htbp]
    \centering
    \resizebox{\textwidth}{!}{
    \begin{tabular}{|C{3cm}|C{3,5cm}|C{1.8cm}|C{2.3cm}|C{2.3cm}|}\hline
        \textbf{Výrobce} & \textbf{Model} & \textbf{Množství} & \textbf{Cena za kus bez DPH} & \textbf{Celková cena bez DPH} \\\hline
        Beckhoff & CX2030-0125 & 2 & \koruna{66650.44} & \koruna{133300.88} \\\hline
        Beckhoff & CX2100-0004 & 2 & \koruna{8663.21} & \koruna{17326.42} \\\hline
        Schneider Electric & BMXP342020 340-20 & 1 & \koruna{30536.8} & \koruna{30536.80}  \\\hline
        Schneider Electric & BMXCPS2000 & 1 & \koruna{5220.27} & \koruna{5220.27}  \\\hline
        Schneider Electric & BMXXBP0400 & 1 & \koruna{3493.03} & \koruna{3493.03}  \\\hline
        Schneider Electric & M251MESC & 2 & \koruna{8192.46} & \koruna{16384.92}  \\\hline
        Schneider Electric & TM262L20MESE8T & 1 & \koruna{36636.86}  & \koruna{36636.86}  \\\hline
        Rockwell Automation & 1769-L24ER-QB1B & 2 & \koruna{38526.54} & \koruna{77053.08} \\\hline
        Rockwell Automation & 1783-BMS20CGN & 2 & \koruna{64480.27} & \koruna{128960.54} \\ \hline
    \end{tabular}
    }
    \caption{Seznam nakoupených zařízení do projektu}
    \label{tab:device_list}
\end{table}


\begin{table}[htbp]
    \centering
    \resizebox{\textwidth}{!}{
    \begin{tabular}{|C{3.5cm}|C{1.8cm}|C{2.7cm}|C{2.7cm}|}\hline
        \textbf{Produkt} & \textbf{Množství} & \textbf{Cena za kus bez DPH} & \textbf{Celková cena bez DPH} \\\hline
        Schneider Electric Machine Expert & 2 & \koruna{23987.18} & \koruna{47974.36}  \\\hline
        SO Machine & 2 & \koruna{26907.28} & \koruna{53814.56}  \\\hline
        Studio 5000 & 2 & \koruna{79140.18} & \koruna{158280.36}  \\\hline
        Unity Pro & 1 & \koruna{102299.92} & \koruna{102299.92}  \\\hline
    \end{tabular}
    }
    \caption{Tabulka licencí potřebných k ovládání nakoupených zařízení}
    \label{tab:device_licenses}
\end{table}
