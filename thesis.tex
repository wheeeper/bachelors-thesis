% arara: xelatex: { shell : yes }
% arara: biber
% arara: xelatex: { shell : yes }
% arara: xelatex: { shell : yes }

% options:
% thesis=B bachelor's thesis
% thesis=M master's thesis
% Czech thesis in Czech language
% Slovak thesis in Slovak language
% English thesis in English language
% hide links remove color boxes around hyperlinks

\documentclass[thesis=B,czech]{template/FITthesis}[2019/12/23]

\usepackage[utf8]{inputenc} % LaTeX source encoded as UTF-8
\usepackage{dirtree} %directory tree visualization

\usepackage{lipsum}
\usepackage{xevlna}
\usepackage{pdfpages}
\usepackage{minted}
\usemintedstyle{friendly}


\usepackage{chngcntr}
\counterwithin{listing}{chapter}

\usepackage[htt]{hyphenat}
\usepackage{caption}

\usepackage[style=iso-numeric]{biblatex}
\addbibresource{ref.bib}

\usepackage{float}

\usepackage{siunitx}
\sisetup{
    output-decimal-marker={,}% just uncomment if you want to use comma as the decimal marker!
}

\def\UrlBreaks{\do\/\do\-\do\_}


\RequirePackage{xcolor} 
\newcommand{\todo}[1]{\textcolor{red}{\textbf{[[#1]]}}}

% % % % % % % % % % % % % % % % % % % % % % % % % % % % % % 
% ODTUD DAL VSE ZMENTE
% % % % % % % % % % % % % % % % % % % % % % % % % % % % % % 

\department{Katedra softwarového inženýrství}
\title{Návrh a implementace knihovny pro automatizaci testů verifikace průmyslové komunikace}
\authorGN{Martin} %(křestní) jméno (jména) autora
\authorFN{Štěpánek} %příjmení autora
\authorWithDegrees{Martin Štěpánek} %jméno autora včetně současných akademických titulů
\author{Martin Štěpánek} %jméno autora bez akademických titulů
\supervisor{Ing.\,{}Miroslav Dušek}
\acknowledgements{Doplňte, máte-li komu a za co děkovat. V~opačném případě úplně odstraňte tento příkaz.}

\abstractCS{Tato bakalářská práce se zabývá návrhem a implementací knihovny pro automatizaci testů verifikace průmyslové komunikace. Vytvořená knihovna funguje jako doplněk do testovacího frameworku MSTest, za pomocí nehož je knihovna následně propojena se serverem Azure DevOps. Ten následně může jednotlivé testy registrovat a automaticky spouštět. Práce také ukazuje open-source knihovny pro průmyslové protokoly ModbusTCP a EtherNet/IP, které jsou vhodné k využití současně s testovací knihovnou, a za pomocí jedné z těchto knihoven následně demonstruje funčknost vytvořeného řešení. Na závěr práce zhodnocuje vytvořené řešení a jeho přínos.} 


\abstractEN{This bachelor's thesis is focused on design and implementation of test automation library for verification of industry fieldbus communication. This library works as add-in to testing framework MSTest, which enables connection with Azure DevOps. Azure DevOps then can register each of the tests and automatically execute them. This thesis also shows open-source libraries for fieldbuses ModbusTCP and EtherNet/IP, which are suitable for use together with the created test library. Thesis also demonstrates the funcionality of the created test library, with the help of one of the open-source library for fieldbus. Lastly the thesis evalutes the created solution and its contribution.}

\placeForDeclarationOfAuthenticity{V~Praze}
\declarationOfAuthenticityOption{4} %volba Prohlášení (číslo 1-6)
\keywordsCS{softwarové testování, automatizace testů, testovací knihovna, verifikace průmyslové komunikace, MSTest, ModbusTCP, EtherNet/IP}
\keywordsEN{software testing, test automation, test library, verification of fieldbus communication, MSTest, ModbusTCP, EtherNet/IP}
% \website{http://site.example/thesis} %volitelná URL práce, objeví se v tiráži - úplně odstraňte, nemáte-li URL práce

\begin{document}

\begin{introduction}
Po industriální revoluci, která uvolnila dělníky z těžké manuální práce, je využití automatizace ve výrobě dalším velkým krokem ve vývoji průmyslu.
Automatizace výroby přináší zefektivnění výroby, zrychlení výroby a snížení nákladů.
Podstatným faktorem k ovládání zařízení, které se starají o automatizaci výroby, je spolehlivá komunikace. Pro tuto komunikace se v dnešní době využívají speciálně vyvinuté průmyslové protokoly.  

Při vývoji těchto zařízení, které se podílí na automatizaci výroby, je stejně důležité jako například jejich návrh také jejich testování. Hlavním úkolem této fáze je odhalení nedostatků produktu, které se liší od dané specifikace produktu. Testováním je tedy produkt kontrolován a z výsledků testů lze odvodit stav a kvalitu produktu. 

V dnešních době se u testování snaží využít výhod automatizace testování. Mezi tyto výhody patří například jednoduchá opakovatelnost testování nebo umožnění častějšího testování. V určitých případech zároveň umožnuje testovat případy, které nelze manuálně testovat.

Tato práce se věnuje návrhu a implementaci knihovny, která bude automatizovat testy verifikace průmyslové komunikace. Hlavní motivací k vytvoření této knihovny je standardizace testování, zjednodušení a zrychlení vytváření testů, což poté vede k šetření času testera. Tato knihovna je vytvářena pro společnost Siemens,~s.\,{}r.\,{}o.

\todo{Dodat co obsahují jednotlivé kapitoly}
\end{introduction}


\chapter{Cíl práce}
Cílem této práce je navrhnout a implementovat knihovnu, která umožní automatizovat testy verifikace průmyslové komunikace. 
Součástí této knihovny má být:
\begin{itemize}
    \item služba, která bude řídit testovací běh,
    \item rozhraní, které umožní implementaci knihovny na testovaném zařízení,
    \item protokol, který bude definovat komunikaci mezi službou a testovanými zařízeními.
\end{itemize}
Vytvořená knihovna poté má být propojena s Azure DevOps serverem tak, aby Azure DevOps server mohl 
následně automaticky spouštět testy. Dalším cílem je provést výzkum dostupných open-source knihoven pro 
průmyslové protokoly ModbusTCP a Ethernet/IP a vybrat vhodné kandidáty na implementaci do této knihovny. 
Součástí této práce má též být sada ukázkových testů pro jeden z průmyslových protokolů, na kterých bude 
vidět demonstrace funkcionality vyvinutého řešení. V neposlední řadě je cílem zhodnotit výsledné řešení z 
pohledu projektového řízení.

\chapter{Teoretická část} % Je to vhodný název kapitoly?

\section{Testování}

Testování je podstatnou součástí vývoje softwaru. Cílem testování není pouze odhalení chyb v softwaru, ale také verifikace a validace softwaru \cite{singh2012software}. Při testování se snažíme vytvářet situace, ve kterých můžeme ověřit, že se software chová dle zadané specifikace.

Testování softwaru je zároveň dovednost \cite{fewster1999software}. Při testování člověk musí vybrat z nekonečného množství možných testů nějaký konečný počet, který nejlépe reprezentuje danou problematiku a pokrývá co největší možnou množinu všech možných případů. Zároveň musí vzít v potaz náročnost na vytvoření testu a na rigidnost vytvořeného testu proti změnám v softwaru. Tyto faktory poté ovlivňují i náklady na testování. Od testování softwaru se také odvodit kvalita softwaru. Kvalitu softwaru se dá určit tím, jak moc vytvořený software odpovídá zadaným specifikacím \cite{software_quality}. Tyto informace jsou poté velmi důležité pro managment a pro další plánování vývoje. 

I když cíl testování je jednotný, přístupů k testování je několik. Vhodnost jednotlivých přístupů se mění na základě testované komponenty. Tyto přístupy se dají rozdělit do několika kategorií \cite{luo2001software}.

\subsection{Podle znalosti komponenty}

Testování se dá rozdělit podle přístupu k informacím, které o komponentách softwaru/systému víme. Tyto typy jsou:

\begin{description}
    \item[Black box testování] Nazýváno taktéž funkční testování. Na software se pohlíží jako na tzv. černou skříňku. O komponentě nebo celku nic nevíme a testujeme na základě funkcionálních požadavků a návrhu. 
    \item[White box testování] Se znalostí implementace testované části se snažíme vytvořit takové testy, které způsobí spouštění určitých částí testované komponenty. Cílem je co největší pokrytí testování dané komponenty.
    \item[Grey box testování] Kombinace Black box a White box testování. Při testování máme nějakou znalost implementace komponenty, ale je nižší, než při White box testování \cite{khan2010different}.
\end{description}

\subsection{Podle částí vývoje}

Testování podle částí vývoje se přibližuje vývojovému cyklu. Tyto kategorie jsou:  

\begin{description}
    \item[Testování částí] V angličtině nazýváno známým pojmem \uv{Unit testing}. Je to nejnižší úroveň testování. Testuje jednotlivé komponenty systému samostatně.
    \item[Integrační testování] Testování dvou a více komponent, které spolu vytváří nějaký větší celek softwaru. Často také využíván při testování částí, které nelze samostatně testovat.
    \item[Systémové testování] Testování softwaru jako celku. Testování se směřuje na testování funkčních požadavků. Zároveň je možno vyhodnocovat další požadavky na systém, jako spolehlivost, bezpečnost, atd.
    \item[Akceptační testování] U toho testování se systém dostane do rukou zákazníkovi/uživatelům. Cílem je otestování produktu u potencionálních uživatelů softwaru a získání jejich zpětné vazby. 
\end{description}

\subsection{Analýza softwaru}
Součástí testování je i analýza softwaru. Tato analýza se dá rozdělit podle toho, zda je potřeba daný vyvíjený software vůbec spouštět. Tyto kategorie jsou:

\begin{description}
    \item[Statická analýza] Tato analýza je prováděno bez spuštění softwaru. Analýza je prováděna na napsaný kód. Vyhodnocovány jsou obecné vlastnosti napsaného kódu bez znalosti kontextu použití.
    \item[Dynamická analýza] Testování je provedeno spuštěním softwaru a použitím reálných metod systému s reálnými daty v simulovaných situacích. 
\end{description}

\section{Automatizace testování}

Automatizace testování je odlišná od samotného testování. Automatizace testu neurčuje samotnou kvalitu testu. Automatizováním testu, který nic nového nepřinese, dostaneme toto nic pouze rychleji \cite{fewster1999software}. Automatizování je ale přesto v mnoha ohledech v dnešní době standardem při testování, a to především díky jeho výhodám. Díky automatizaci jsme při vývoji schopni provádět opakované testy za frakci ceny, než kdyby byli prováděny manuálně. Toto zároveň uvolňuje testery ke směřování své snahy k rozšiřování množiny testů a tím pokrytí co nejvíce případů.

\todo{Rozšíření textu o automatizaci}


\section{Průmyslová komunikace}

Při řešení průmyslové komunikace se často objevuje slovo \textit{fieldbus}. Běžný význam tohoto slova je \uv{Síť, která propojuje průmyslová zařízení jako kontrolery, PLC, regulátory atd.} \cite{fieldbus_thomesse}. Vznik těchto sítí je úzce spojený s historií vývoje informačních technologií. V době, kdy začali tyto sítě vznikat, nebyli dostupné komunikační technologie, jako dnes. Dostupné informační a telekomunikační sítě té doby nemohli uspokojit potřeby průmyslových sítí na deterministickou, spolehlivou a efektivní komunikaci \cite{future_of_ind_com}. 

V dnešní době jsou tyto průmyslové sítě mezinárodně standardizovaný. Jako příklad protokolů můžeme uvést ModbusTCP, nebo Ethernet/IP, které už využívají výhod Ethernet připojení a zároveň satisfakují průmyslové potřeby. 


\chapter{Návrh}\label{chap:design}

V této kapitole se zabývám návrhem testovací knihovny a jejími funkčními požadavky.

\section{Účastníci testování}\label{sec:participants}
V rámci knihovny a testovacího běhu se bude vyskytovat několik účastníků testování. Mezi účastníky testování bude patřit:

\begin{description}
    \item[Testovací služba] Služba, která řídí testovací běh.
    \item[Testované zařízení] Hlavní účastník testování, který běží na jiném zařízení, než ze kterého běží testovací služba. 
    \item[Testovací partner] Zařízení, které simuluje nějaké testované zařízení. Toto zařízení běží na stejném zařízení, jako testovací služba. 
\end{description}

Testovací služba bude jádrem k řízení testování. Poběží na samostatném zařízení, které bude běžet na operačním systému Windows. Zařízení bude v~rámci infrastruktury serveru Azure DevOps agent. Testovací služba bude implementována v jazyce \csharp{}. Všechna implementace v~jazyce \csharp{} bude vytvořena pro .NET Framework 4.8. 

Hlavním cílem je testovat vyvíjený produkt. Toto zařízení bude v rámci knihovny bráno jako testované zařízení. Testovací služba bude podporovat připojení jednoho a více testovaných zařízení a při každém testovacím běhu musí být připojeno alespoň jedno testovací zařízení. Tato zařízení budou připojena k zařízení, ze kterého poběží testovací služba, za pomoci Ethernet připojení. Zařízení SIMATIC ET 200SP má svoji implementaci v jazyce \cpp{}. Z~toho vyplývá, že implementace pro toto zařízení bude vytvořena v tomto jazyce. 

Součástí knihovny zároveň bude také tzv. testovací partner. Tento účastník slouží k simulaci testovaného zařízení, buď jako celku, nebo pouze nějaké jeho činnosti. Zařízení může simulovat zařízení jako PLC, nebo různé nástroje, které slouží například k certifikaci vyvíjeného zařízení. Simulováním zařízení snižujeme hardwarové nároky na testování, což vede ke snížení ekonomických nákladů na testování. Tato zařízení se budou připojovat v závislosti na jednotlivých testech. Následně po skončení testu budou tato zařízení odpojena. Testovací zařízení bude stejně jako testovací služba vytvořeno v jazyce \csharp{}.

Testovací služba bude očekávat připojení minimálně jednoho testovaného zařízení. Následně maximální počet připojených zařízení ke službě se bude odvíjet od limitací protokolu TCP/IP a jeho implementace v jazyce \csharp{}. Tento limit se ale pohybuje ve stovkách, v kontrastu s tím se očekávaný počet zároveň připojených zařízení bude pohybovat v desítkách. Tedy tento limit knihovnu nijak neomezuje.

Účastníky testování a jejich možné propojení znázorňuje obrázek \ref{fig:devicemodel}. Zde můžeme vidět agenta, resp. zařízení, na kterém poběží testovací služba a případní testovací partneři, primární testované zařízení SIMATIC ET 200SP a~PLC. PLC zařízení značí jakékoliv zařízení, které není simulováno knihovnou.

\begin{figure}[htbp]
    \centering 
    \includegraphics[width=0.97\textwidth]{assets/img/devicemodel.pdf}
    \caption{Ukázka možného propojení účastníků testování}
    \label{fig:devicemodel}
\end{figure}

\section{Komunikace}\label{sec:communication}
Komunikace se všemi účastníky testování bude fungovat na principu TCP/IP připojení. Testovací služba a zařízení, na kterém služba poběží, bude sloužit jako server a všichni účastníci testovaní budou klienti, kteří se budou k tomuto zařízení připojovat. 

Jednotlivé zprávy vyměňované mezi testovací službou a všemi účastníky testování budou mít jasně stanovenou strukturu. Diagram složení zprávy lze vidět na obrázku \ref{fig:message}, z něhož je patrné, že jedna zpráva lze rozdělit na dvě části -- hlavičku zprávy a data zprávy. Zároveň, jak můžeme vidět, jedna buňka v diagramu odpovídá jednomu bajtu. 

\begin{figure}[htbp]
    \centering 
    \includegraphics{assets/img/message.pdf}
    \caption{Diagram struktury jedné zprávy}
    \label{fig:message}
\end{figure}

Zpráva bude povinně obsahovat hlavičku zprávy, kde bude uveden typ zprávy a délka dat zprávy. Jak můžeme vidět na diagramu, každý z~těchto údajů bude mít velikost dva bajty. Celkově tedy hlavička bude o velikosti 4~bajty. Knihovna bude podporovat tyto typy zpráv:

\begin{enumerate}
    \item \inlinecode{MSG\_OK} -- Zpráva o úspěchu/potvrzení
    \item \inlinecode{MSG\_FAIL} -- Zpráva o neúspěchu
    \item \inlinecode{MSG\_TEST} -- Direktiva ke spuštění testu
    \item \inlinecode{MSG\_STOP} -- Direktiva k ukončení testování
\end{enumerate}

Za hlavičkou zprávy následně budou moct být uložena data zprávy. Tato data však budou nepovinná. V těchto případech, kdy zpráva nebude neobsahovat data zprávy, bude pro přenesení informace postačovat pouze typ zprávy a v hlavičce bude uvedena nulová hodnota jako délka dat zprávy. 

Maximální délka jedné zprávy se bude odvíjet od limitací použitých technologií. Limit se bude primárně odvíjet od Ethernet protokolu, který podporuje nejmenší maximální délku jednoho paketu ze všech použitých protokolů, a to 1500 bajtů. Toto číslo také zahrnuje všechny potřebné hlavičky protokolu TCP/IP k přenosu paketu. Z toho důvodu bude jako maximální délka jedné zprávy zvolena délka 1400 bajtů. Toto číslo ponechává místo pro potřebné hlavičky a zároveň je více než dostatečné pro navrhované použití. \cite{max_packet_size} 

Všechny hodnoty zprávy budou ukládány v kladné (anglicky tzv. unsigned) podobě. Jedinou kontrolní informací, kterou bude zpráva obsahovat, je délka dat zprávy. Protokol TCP/IP sám obsahuje jednoduchou detekci chyb. Dohromady tyto kontroly budou považovány jako dostatečné.

Účastníci testování se na začátku připojí k testovací službě a odešlou inicializační zprávu. Tato zpráva bude typu 1 a v datech zprávy bude obsahovat svou MAC adresu. Výjimkou budou testovací partneři, kteří budou jako svoji MAC adresu odesílat adresu \inlinecode{DE:AD:BE:EF:00:00}. Díky tomu je testovací služba jednoduše identifikuje. Testovací služba následně odešle potvrzovací zprávu o úspěšném připojení s typem 1, bez žádných dat. V případě vzniku chyby služba odešle zprávu typu 2, bez dat. 

Pro započetí testování a spuštění jednotlivých stádií testování služba odešle všem účastníkům testování zprávu s typem 3. V datech zprávy služba odesílá tato data:

\begin{enumerate}
    \item Číselnou reprezentaci identifikátoru testu. Ten je uložen v prvních 4~bajtech dat zprávy.
    \item Číselnou reprezentaci identifikátoru, který určuje fázi testu. Uložen je ve dvou bajtech dat zprávy hned za identifikátorem testu. Fáze testování jsou blíže popsány v sekci \ref{sec:test_run}.
\end{enumerate}

Služba následně čeká na odpověď od všech účastníků testování. Účastnící odesílají testovací službě po skončení testovacího stádia zprávu bez dat, s~typem 1 v případě úspěchu a s~typem 2 v případě neúspěchu. Po skončení testování služba odesílá všem účastníkům testování zprávu s~typem 4, bez žádných dat. 

Ukázka komunikace je znázorněna na sekvenčním diagramu, viz obrázek \ref{fig:seqdiag}. Na tomto sekvenčním diagramu můžeme vidět výměnu zpráv mezi testovací službou a dvěma účastníky testování během spuštění jednoho testu. Direktiva \inlinecode{Message} znázorňuje jednu zprávu, kde v závorkách můžeme vidět nejdříve typ zprávy a následně data zprávy (pokud zpráva nějaká data obsahuje). Rovněž lze v diagramu spatřit využití testovacího partnera.

\begin{figure}[htbp]
    \centering 
    \includegraphics[width=\textwidth]{assets/img/sequencediagram.pdf}
    \caption{Sekvenční diagram ukázky komunikace mezi účastníky testování}
    \label{fig:seqdiag}
\end{figure}

\section{Testovací služba}
Jak jsem již zmínil v sekci \ref{sec:participants}, testovací služba bude jádrem celého testování. Tato služba bude řídit celý testovací běh a předávat všem účastníkům testování pokyny. Zároveň bude synchronizovat testovací běh mezi všemi účastníky testování. Běh služby znázorněný na diagramu aktivit vidíme na obrázku \ref{fig:activitydiagramservice}. Jednotlivé aktivity jsou vysvětleny v následujících sekcích. 

\begin{figure}[htbp]
    \centering 
    \includegraphics[width=\textwidth]{assets/img/activitydiagramservice.pdf}
    \caption{Diagram aktivit testovací služby}
    \label{fig:activitydiagramservice}
\end{figure}

\subsection{Inicializace služby}
Testovací služba začíná běh inicializační fází. Služba v inicializační fázi vytvoří připojení se všemi testovanými zařízeními dle nadefinované komunikace. Z konfigurace bude služba vědět, kolik testovaných zařízení má očekávat. Inicializační fáze zároveň bude mít definovaný časový limit, který ve výchozí konfiguraci bude 60 sekund. Zároveň ale bude nastavitelný uživatelem. Testovací služba vyhodnotí inicializační fázi jako úspěšnou, pokud se úspěšně připojí definovaný počet testovaných zařízení. Služba vyhodnocuje neúspěch inicializační fáze v případě, že nastanou tyto situace:

\begin{itemize}
    \item vypršení časového limitu na inicializační fázi,
    \item chyba v komunikaci, nesprávná komunikace,
    \item nepřipojení definovaného počtu zařízení,
    \item připojení testovacího partnera.
\end{itemize}

V případě neúspěchu inicializační fáze služba nastavuje stav služby jako chybný. Testy poté nebudou provedeny.

\subsection{Spravování testovacích partnerů}
Služba bude podporovat připojení testovacích partnerů. Tato zařízení se budou připojovat před spuštěním jednotlivých testů. Testovací služba obdrží direktivu k očekávaní připojení testovacího partnera. Partner následně projde stejnou inicializační fází jako ostatní testovaná zařízení. Služba vyhodnocuje neúspěch připojení testovacího partnera v těchto případech:

\begin{itemize}
    \item připojení jiného zařízení, než testovacího partnera,
    \item vypršení časového limitu na připojení,
    \item chyba v komunikaci.
\end{itemize}

V opačném případě služba vyhodnocuje úspěch. Po dokončení testu služba odesílá všem připojeným partnerům direktivu k ukončení testování a ukončuje spojení s nimi. 

\subsection{Registrování testů}
Testovací služba nebude mít žádné informace o jednotlivých testech. Jedinou informaci, kterou testovací služba obdrží, bude číselný identifikátor testu. Reprezentace testů za pomocí čísel může způsobit kolizi mezi testovými identifikátory. Služba samotná tuto kolizi kontrolovat nebude. Způsob detekce kolize je popsán v sekci \ref{sec:reg_test_design}.

\subsection{Testovací běh}\label{sec:test_run}
Po úspěšné inicializační fázi a případném připojení testovacích partnerů služba čeká na direktivu ke spuštění testu. Po obdržení této direktivy s identifikátorem testu služba odesílá všem účastníkům testování zprávu k zahájení první fáze testu. Tuto fázi budeme označovat jako přípravu na testování. V této fázi účastníci testovaní připraví všechny potřebné prostředky pro provedení testu. Účastníci následně odesílají zprávu o úspěchu/neúspěchu této fáze. Služba vyhodnocuje fázi jako úspěšnou, pokud od všech účastníků obdrží zprávu o~úspěchu. V opačném případě vyhodnocuje fázi jako neúspěšnou.

Služba po vyhodnocení úspěchu první fáze přechází do fáze druhé. V této fázi proběhne samotné testování. Služba odešle všem účastníkům zprávu ke spuštění této fáze. Následně očekává odpověď od všech účastníků. Služba, stejně jako v předchozí fázi, vyhodnocuje fázi jako úspěšnou, pokud obdrží zprávu o úspěchu od všech účastníků testování. V opačném případě je neúspěšná.

Poslední fází testu je úklid po testu. V této fázi služba opět vyšle zprávu k započetí fáze. Účastníci testování v této fázi uvádějí zařízení do stavu, ve kterém bylo před zahájením testu. Následně účastníci odešlou zprávu o úspěchu/neúspěchu. Vyhodnocení úspěchu/neúspěchu je stejné jako v předchozích fázích.

Po odeslání zprávy s direktivou ke spuštění jednotlivých fází testu služba čeká, než obdrží odpověď od všech účastníků testu. Tyto body, kdy služba čeká na odpověď od všech účastníků testu, budeme nazývat synchronizačními body. Účastníci, kteří dokončili svojí fázi testu, čekají na direktivu od testovací služby. Tímto se běh synchronizuje mezi všemi účastníky testu. Jednoduše lze odvodit, že během jednoho testu nastávají tři synchronizační body. 

Tyto synchronizační body budou mít definovaný časový limit. Při spuštění testu bude tester moci zadat vlastní časový limit, který bude použit po každé z~fází. Tedy tester bude zadávat předpokládaný maximální časový limit nejdelší fáze testování. Zároveň je ale potřeba vzít v úvahu, že časový limit je spuštěn ihned po odeslání všech zpráv. Tento časový limit bude ve výchozí konfiguraci opět 60 sekund. 

Testovaná zařízení se v době běhu testu mohou dostat do chybového stavu, ve kterém nebude možné pokračovat v testování. Testovací služba tedy bude předpokládat tyto chybné stavy:


\begin{description}
    \item[Nezajištění konzistence testování] Pro konzistenci testů testovací služba vyžaduje, aby po každém testu testované zařízení bylo ve stejném stavu jako před jeho začátkem. Pokud účastník testu odešle neúspěch z důvodu vzniklé chyby ve fázi přípravy na test, nebo ve fázi úklidu po testu, tak poté bude tento účastník považován jako v chybném stavu. V případě, že chyba nastane ve fázi přípravy na test, služba neprovádí fázi testování a přechází do fáze úklidu po testu. 
    \item[Vypršení časového limitu] Vypršení časového limitu primárně znamená, že služba ve stanové době neobdržela odpověď od účastníka testování, ale tento účastník je stále připojen k testovací službě. Toto vede k závěru, že tento účastník je v chybovém stavu a další testování není možné. 
    \item[Chyba v komunikaci] Pokud během testu zařízení odpoví jinak, než dle stanovené komunikace, bude toto zařízení považováno jako by bylo v chybném stavu. Zároveň bude za chybu považováno odpojení zařízení od služby.  
\end{description}

Pokud všechny fáze testu byly úspěšné, služba poté vyhodnocuje test jako úspěšný. Opačně pak služba vyhodnocuje test jako neúspěšný, pokud jedna z~fází byla neúspěšná. V~případě nastání chyby, kde testované zařízení je předpokládáno jako v~chybovém stavu, testovací služba nastavuje svůj stav jako chybový. Následující testy nebudou provedeny.


\subsection{Ukončení testování}
Testovací služba po obdržení direktivy k ukončení služby odesílá všem připojeným účastníkům testování zprávu k ukončení testovacího běhu. I když služba může některého z účastníků považovat jako by byl v chybném stavu, pokud je tento účastník stále ke službě připojen, služba mu odešle tuto ukončovací zprávu. Toto se děje s cílem úspěšně ukončit co nejvíce zařízení.


\section{Rozhraní pro testování}
Důležitou součástí knihovny budou rozhraní, která umožní implementaci propojení s testovací službou a vytváření jednotlivých testů. Knihovna bude obsahovat dvě důležitá rozhraní. 

\subsection{Rozhraní testu}
Jak jsem již popsal v sekci \ref{sec:test_run}, jednotlivé testy budou mít tři fáze testování:

\begin{enumerate}
    \item Příprava na testování -- definování potřebných struktur, inicializace.
    \item Testování -- provedení samotného testu.
    \item Úklid po testu -- uvolnění využitých zdrojů a uvedení zařízení do původního stavu.
\end{enumerate}

Testovací knihovna tedy bude definovat rozhraní, které bude vyžadovat, aby tyto tři fáze testu byly pro každý test definované. Návrh rozhraní můžeme vidět na výpisu \ref{listing:test_if}, kde každé fázi odpovídá jedna metoda. Tyto metody následně vrací jednoduchou informaci o úspěchu/neúspěchu. To znamená, že vyhodnocení jednotlivých fází bude na jednotlivých testerech a jejich implementaci. Jednotlivé komponenty knihovny pouze obdrží informaci o výsledku fáze.

\begin{listing}[htbp]
    \centering
    \begin{cminted}[breaklines]{text}
ITestCase 
{
    bool StartUp()

    bool Test()

    bool TearDown()
}
    \end{cminted}
\caption{Návrh rozhraní pro jeden test}
\label{listing:test_if}
\end{listing}


\subsection{Rozhraní pro testované zařízení}\label{sec:deviceif}

Dalším důležitým rozhraním bude rozhraní pro testované zařízení. Toto rozhraní bude definovat metody, které bude potřeba definovat na každém testovaném zařízení. Cílem je, aby toto rozhraní bylo co nejjednodušší pro co nejrychlejší implementaci na nově testovaném zařízení.

Návrh tohoto rozhraní můžeme vidět na výpisu \ref{listing:dev_if_abstract}. Jak si můžeme všimnout, první tři metody se starají o propojení s testovací službou. Metoda \inlinecode{createConnection} bude implementovat vytvoření propojení s testovací službou. Metoda pouze vytvoří propojení s testovací službou na základě TCP/IP protokolu a následně vrátí informaci o úspěšnosti. Cíl připojení, tedy IP adresu a port, na kterém poběží testovací služba, metoda obdrží v argumentech.

O samotné odesílání, resp. přijímání jednotlivých zpráv se bude starat metoda \inlinecode{sendMessage}, resp. \inlinecode{rcvMessage}. Metoda \inlinecode{sendMessage} po svém zavolání převede objekt zprávy, obdržený v argumentu metody, na bajtové pole, které následně odešle testovací službě. Metoda \inlinecode{rcvMessage} symetricky zprávu od testovací služby přijme a převede ji na objekt jedné zprávy. Následně odkaz na tuto zprávu uloží do argumentu, který je výstupní. Obě tyto metody vrací informaci o úspěchu/neúspěchu.

Metoda \inlinecode{getTest} bude určena k získání jednotlivých instancí testů. Testy, které bude tato metoda vracet, musí být vytvořeny dle dříve nadefinovaného rozhraní pro jednotlivé testy. Metoda obdrží číselnou reprezentaci testu a na základě toho poté vrátí správnou instanci testu. 

Metoda \inlinecode{getMacAddress} bude, jak již z názvu vyplývá, vracet MAC adresu zařízení. Metoda \inlinecode{print} bude sloužit k výpisu průběhu běhu. Tento výpis by měl hlavně posloužit k náhledu na průběh testovacího běhu, pokud se během něj vyskytnou například nějaké chyby. Nakonec metoda \inlinecode{stop} slouží k ukončení běhu zařízení.


\begin{listing}[htbp]
    \centering
    \begin{cminted}[breaklines]{text}
ITestClient 
{
    bool createConnection(ipAddress, port)

    bool sendMessage(message)

    bool rcvMessage(message)

    ITestCase getTest(test)

    void stop()

    void print(toPrint)

    bool getMacAddress(macAddr)
}
    \end{cminted}
\caption{Návrh rozhraní pro testované zařízení}
\label{listing:dev_if_abstract}
\end{listing}

\section{Běh testovaného zařízení}

O samotný běh jednotlivého testovaného zařízení se bude starat samostatná komponenta, která bude nezávislá na jakémkoliv zařízení. Tato komponenta se bude nazývat v knihovně tzv. \inlinecode{TestRunner}.

Komponenta bude obsahovat veškerou logiku testovacího běhu pro jednotlivé testované zařízení. Při zahájení testování komponenta za pomocí rozhraní vytvoří propojení s testovací službou a odešle zprávu o úspěchu, kde v datech zprávy uvede svoji MAC adresu. Po obdržení zprávy o úspěchu od testovací služby komponenta přechází do metody, která bude obsluhovat příchozí zprávy. 

Komponenta bude očekávat dva typy zpráv - direktivu ke spuštění testu a~direktivu k ukončení testování. Po obdržení direktivy k započetí testu, komponenta ze zprávy zjistí identifikátor testu a testovací fázi. Následně z rozhraní získá instanci testu, který si po dobu běhu testu drží. Poté dle instrukcí spouští jednotlivé fáze testování. Po konci každé testovací fáze komponenta odesílá zprávu o úspěchu/neúspěchu testovací službě.

Dle této implementace může teoreticky dojít k tomu, že testované zařízení obdrží zprávu s požadavkem na spuštění fáze testování nebo fáze úklidu po testu ještě předtím, než obdrží instrukci ke spuštění fáze přípravy na test. Komponenta tuto situaci kontroluje a v případě jejího vzniku automaticky odesílá zprávu o neúspěchu fáze. 

Průběh testovacího běhu testovaného zařízení lze vidět na diagramu aktivit na obrázku \ref{fig:act_diag_device}. Jak lze z diagramu vyčíst, běh zařízení se silně odvíjí od příkazů, které obdrží od testovací služby. Zároveň na diagramu můžeme vidět již zmíněné tři synchronizační body. Tyto body jsou zvýrazněny modrým čárkovaným obdélníkem. Z diagramu se může zdát, že pokud zařízení ve fázi přípravy na testování nebo ve fázi úklidu po testu vrátí neúspěch, lze poté přesto pokračovat v testování. Toto ale není pravda, protože tento případ testovací služba nepovolí a po dokončení testu, který způsobil danou chybu, bude testovací běh ukončen. Výjimkou jsou pouze testovací partneři. Tato zařízení jsou po konci testu ukončována, proto jejich chybný stav nemá následky na další testy.

\begin{figure}
    \centering 
    \includegraphics[height=0.98\textheight]{assets/img/activitydiagramdevice.pdf}
    \caption{Diagram aktivit testovaného zařízení}
    \label{fig:act_diag_device}
\end{figure}


\section{Testovací partner}
Testovací partner je speciálním druhem účastníka. Toto zařízení bude spouštěno ze stejného zařízení jako testovací služba. Primárně bude fungovat na stejném principu jako testovaná zařízení. Zařízení tedy bude mít implementované rozhraní pro testované zařízení, které bude ovládáno komponentou \inlinecode{TestRunner}. Testovací partner ale bude také podporovat použití jakéhokoliv rozhraní pro testované zařízení. 

Změnou je způsob ovládání této komponenty. Komponenta bude zaobalena vlastní třídou, která tuto komponentu bude spouštět na vlastním vlákně. Toto bude umožňovat nezávislý běh zařízení. Rozdílem také bude způsob registrování testů. Testovací partner bude podporovat dva typy způsobů, jak registrovat testy. První způsobem bude za pomocí předání jedné instance testu. Jelikož testovací partner je vázán na jednotlivé testy, je tento přístup nejjednodušší. Dalším způsobem bude předání funkce, která bude vracet instance testů na základě číselného identifikátoru. Tento způsob je totožný jako u testovaných zařízení a v určitých případech může usnadnit správu těchto zařízení. 

Testovací partner bude spouštěn z testovacího projektu za pomoci testovací knihovny. Vytvořená instance partnera bude předána testovací knihovně, která se následně postará o jeho spuštění a ukončení. Zařízení bude po konci testu odpojeno od testovací služby. Testovací knihovna bude kontrolovat úspěšnost tohoto ukončení a v případě neúspěchu standardního ukončení testovací knihovna ukončí vlákno, na kterém zařízení poběží.

Testovací služba nebude nijak omezovat maximální možný počet připojených testovacích partnerů. Jediné omezení, které teoreticky může vzniknout, bude v případě, že při vytváření vláken pro jednotlivá zařízení se vyčerpá maximální počet dostupných vláken. Toto by ale při reálném použití neměl být problém, jelikož je předpokládáno, že počet těchto zařízení využitých pro jeden test se bude pohybovat maximálně v desítkách. 

\clearpage

\section{Propojení se serverem Azure DevOps}

Jedním z cílů této práce je vytvoření takového připojení, aby server Azure DevOps byl schopen spouštět jednotlivé testy a zároveň obdržel výsledek testu. K tomuto propojení využijeme testovací framework MSTest, který je nativně podporovaný serverem Azure DevOps. Celé následující propojení je možné díky uložení projektu v Azure Repos.

\subsection{Propojení s frameworkem MSTest}
Cílem vytvořeného propojení je, aby bylo co nejvíce automatizované a obsahovalo co nejméně práce ze strany testera. MSTest bude nezávisle na jednotlivých testech inicializovat službu před započnutím testování. V této době proběhne inicializační fáze služby. 

Následně framework MSTest bude spouštět jednotlivé vybrané testy. V~testech bude zadávat testovací službě direktivu ke spuštění testu. Zároveň bude moci před zavoláním direktivy přidávat jednotlivé testovací partnery. Následně po skončení testu framework MSTest vyhodnotí úspěšnost testu.

Po proběhnutí jednotlivých testů framework MSTest službu ukončuje. Pokud vzniknou chyby během testování, díky kterým nebude možné pokračovat v testování, bude framework vyhodnocovat následující testy jako bezvýsledné a testy přeskočí.

Propojení testovací služby a testovacího frameworku MSTest bude zajišťovat API. Toto API bude určovat přístupné metody, které mohou být využity v testovacím frameworku. Zároveň bude jedinou komponentou, kde bude framework MSTest v knihovně využívat.

\subsection{Registrování testů}\label{sec:reg_test_design}
Registrování testů ve frameworku MSTest a testovací službě bude na sobě nezávislé. Framework MSTest automaticky objevuje metody, které jsou označeny jako testovací. Framework MSTest v~implementaci pro jazyk \csharp{} využívá pro rozlišení testovacích součástí tzv. atributy. Atributy přidávají metadata ke třídám, typům, metodám atd. \cite{attribute_docs}. Testovací knihovna bude rozšiřovat tyto atributy o atribut \inlinecode{TestEnum}. Tento atribut bude identifikovat typy enumerátorů, které reprezentují testy. 

Tím, že testovací služba může přijmout jakýkoliv enumerátor, může vzniknout kolize mezi jednotlivými typy enumerátorů. V inicializační fázi proto bude API kontrolovat, zda enumerátory s atributem \inlinecode{TestEnum} mezi sebou neobsahují kolizi a v případě jejího vzniku vyhodí výjimku a nepokračuje v~testování.

Jednotlivé testy frameworku MSTest následně budou registrovány na serveru Azure DevOps. Na tomto serveru tester vytvoří tzv. Test Case, který následně asociuje odpovídající testovací metodě frameworku MSTest. Vytvoření této asociace je nativně podporované v IDE Visual Studio 2019, v němž je celá knihovna vyvíjena. 


\subsection{Testování}
Spuštění testů na serveru je možné díky propojení Azure Pipelines a Azure Test Plans. Azure Pipelines stáhne testovací projekt na agenta z Azure Repos a následně ho dle direktiv zkompiluje. Server následně na agentovi spouští testovací projekt a spouští testy definované v určeném testovacím plánu. Tyto plány, vytvořené v Azure Test Plans, shlukují jednotlivé registrované testy a~určují, které testy mají být spuštěny. Po dokončení testů server Azure DevOps automaticky obdrží výsledky testů. Tyto výsledky jsou následně dohledatelné u jednotlivých testů v Azure Test Plans a u jednotlivých běhů v Azure Pipelines. 


\section{Distribuce knihovny}

Knihovna bude distribuována jako NuGet balíček. Zapouzdření knihovny do NuGet balíčku umožní jednoduché verzování, distribuci a instalaci. Zároveň NuGet balíček vytváří asociace k ostatním NuGet balíčkům, které budou použity v knihovně.

Tento NuGet balíček bude následně nahrán do služby Azure Artifacts, jejímž využitím udržíme stejnou jednoduchost distribuce NuGet balíčku a zároveň limitujeme veřejný přístup k tomuto balíčku. 

Na obrázku \ref{fig:deploymodel} můžeme vidět distribuci jednotlivých částí knihovny, kam jsou její jednotlivé části distribuovány a jak je rozdělena. Součástí dříve zmiňovaného NuGet balíčku bude celá implementace v jazyku \csharp{}, což zahrnuje testovací službu, testovacího partnera a rozhraní pro testovaná zařízení v jazyku \csharp{}. 

\begin{figure}[p!]
    \centering 
    \includegraphics[width=0.95\textwidth]{assets/img/deploymentmodel.pdf}
    \caption{Model distribuce jednotlivých částí knihovny}
    \label{fig:deploymodel}
\end{figure}

\clearpage

\section{Využití průmyslových protokolů}
Jak jsem již definoval v sekci \ref{sec:fieldbus}, protokoly EtherNet/IP a ModbusTCP mají jasně stanovenou komunikaci. Jejich implementace však není triviální. Z tohoto důvodu je tedy vhodné využít již vytvořené open-source knihovny, které implementují tyto protokoly. V následujících sekcích ukáži vhodné knihovny, které mohou být využity v testovací knihovně.

\subsection{Průzkum open-source knihoven}
K tomu abych mohl správně ohodnotit dostupné open-source knihovny je potřeba definovat kritéria, která musí knihovna splňovat. Jednotlivé knihovny musí být kompatibilní s vytvořenou testovací knihovnou. Tedy, knihovny by měly být implementovány v jazyce \csharp{} a kompatibilní s .NET Framework 4.8. 

Vybrané knihovny budu hodnotit na základě několika faktorů. Mezi tyto faktory bude patřit:

\begin{itemize}
    \item Aktivita autora/autorů, hodnocená na stupnici 1--10, kde 1 značí nízkou aktivitu autorů a 10 vysokou aktivitu.
    \item Náročnost využití knihovny, tedy náročnost implementace do knihovny a náročnost následného použití knihovny, hodnocené na stupnici 1--10, kde 1 značí nejjednodušší a 10 nejtěžší.
    \item Hodnocení dokumentace na stupnici 0--10, kde 0 značí, že knihovna neobsahuje žádnou dokumentaci a 10 značí nejkvalitnější dokumentaci.
    \item Licenční restrikce a možnosti využití knihovny v komerčním prostředí.
\end{itemize}

Následně na základě těchto kritérií a mém osobním názoru doporučím knihovnu vhodnou k použití s testovací knihovnou. 

\subsubsection{ModbusTCP}

V tabulce \ref{tab:modbus} můžeme vidět seznam nejvhodnějších ModbusTCP knihoven pro využití v knihovně. Dle jednotlivých hodnocení je vidět, že knihovny se pohybují okolo stejné náročnosti na využití v knihovně. Všechny jsou dostupné jako NuGet balíčky, což jejich implementaci velice usnadňuje. Všechny knihovny zároveň obsahují srovnatelnou sadu funkcí.

Nejpoužívanější knihovnou, dle počtu stažení NuGet balíčku, je knihovna NModbus. Má průměrnou dokumentaci, která popisuje jednotlivé třídy. Tato knihovna je udržována komunitně, a již dvakrát byla opuštěna předchozími autory. V kontextu toho, že knihovna má nejstarší datum poslední aktualizace a žádnou viditelnou aktivitu, zde vzniká otázka, zda bude v budoucnu udržována.

Knihovna Modbus je opět komunitním projektem. I když jsou autoři vysoce aktivní, knihovna bohužel neobsahuje, až na krátkou ukázku použití, žádnou dokumentaci. Proto má knihovna v hodnocení zvýšenou náročnost na využití díky této chybějící dokumentaci.  

Knihovna FluentModbus je vytvářena jednotlivcem. Knihovna obsahuje dle mého hodnocení nejlepší dokumentaci, avšak jako jediná je vydávána pod LPGL licencí, oproti ostatním knihovnám, které jsou vydávány pod licencí MIT.

Jako nejvhodnější knihovnu jsem vyhodnotil EasyModbusTCP. Knihovna je jako jediná vytvořena komerčním subjektem, a to firmou Rossmann Engineering. Z tohoto důvodu předpokládám, že na knihovnu jsou vyvíjeny vyšší kvalitativní nároky, než na ostatní knihovny. 

Vybraná knihovna splňuje všechny kvalitativní kategorie a zároveň je vydávána pod MIT licencí, která je méně omezující než licence LPGL.

\subsubsection{EtherNet/IP}

Dostupných knihoven pro protokol EtherNet/IP je méně, než pro protokol ModbusTCP. Seznam dostupných knihoven můžeme vidět v tabulce \ref{tab:ethernet}. Výhodou opět je, že všechny knihovny jsou koncipovány jako NuGet balíčky. Knihovny EthernetIP a Incore bohužel neobsahují žádnou dokumentaci a obě obdržely poslední aktualizaci v roce 2018. Náročnost využití je tedy kvůli chybějící dokumentaci výrazně zvýšena. Knihovna EthernetIP také neobsahuje žádné licenční informace, oproti tomu knihovna Incore je vydána pod licencí MIT.

Nejlépe opět vyšla knihovna od firmy Rossmann Engineering, neboť jako jediná obsahuje kvalitní dokumentaci s ukázkami použití knihovny. Zároveň je vydána pod licencí MIT, díky které ji lze použít v komerčním prostředí. Proto jsem tuto knihovnu vyhodnotil jako nejvhodnější pro použití v testovací knihovně.

\begin{table}[p!]\centering
    \resizebox{\textwidth}{!}{
        \begin{tabular}{|C{3cm}|C{3cm}|C{1.8cm}|C{1.8cm}|C{2.3cm}|C{4cm}|}\hline
            Název & Autor & Hodnocení aktivity & Náročnost využití & Hodnocení dokumentace &  Dostupné na adrese \\\hline 
            EasyModbusTCP & Rossmann Engineering & 6 & 1 & 5 & \url{www.easymodbustcp.net}\\\hline
            FluentModbus & Apollo3zehn & 7 & 1 & 6 & \url{www.github.com/Apollo3zehn/FluentModbus} \\\hline 
            NModbus & Rich Quackenbush, et al. & 4 & 2 & 4 & \url{www.github.com/NModbus/NModbus} \\\hline 
            Modbus & Andres Müller, et al. & 8 & 3 & 1 & \url{www.github.com/AndreasAmMueller/Modbus} \\\hline
        \end{tabular}
    }
    \caption{Seznam dostupných knihoven pro protokol ModbusTCP}
    \label{tab:modbus}
\end{table}

\begin{table}[p!]\centering
    \resizebox{\textwidth}{!}{
        \begin{tabular}{|C{3cm}|C{3cm}|C{2.1cm}|C{1.8cm}|C{2.3cm}|C{4.7cm}|}\hline
            Název & Autor & Hodnocení aktivity & Náročnost využití & Hodnocení dokumentace &  Dostupné na adrese \\\hline 
            EEIP & Rossmann Engineering & 5 & 1 & 5 & \url{www.eeip-library.de/}\\\hline
            EthernetIP & SecondShiftEngineer & 	2 & 4 & 0 & \url{www.nuget.org/packages/EthernetIP/} \\\hline
            Incore & Yanjun Wang & 	2 & 4 & 0 & \url{www.nuget.org/packages/inc.protocols.ethernetip/}\\\hline
        \end{tabular}
    }
    \caption{Seznam dostupných knihoven pro protokol EtherNet/IP}
    \label{tab:ethernet}
\end{table}

\chapter{Implementace}

Tato kapitola se věnuje implementační části všech navrhnutých komponent, které umožnují automatizaci testování.

\section{Komunikace}
Testovací knihovna má jasně definovanou komunikaci. Strukturu jedné zprávy zajišťuje třída \inlinecode{Message}. Dle návrhu má jedna odeslaná zpráva definované tři atributy -- délku zprávy, typ zpráva a data zprávy. Třída \inlinecode{Message} si drží dva z těchto atributů, třetí -- délka zprávy -- je na základě ostatních dvou vypočítána. Typ zprávy je určen enumerátorem \inlinecode{MessageType}. Tento enumerátor obsahuje výčty:

\begin{itemize} 
    \item \inlinecode{MSG\_FAIL} -- zpráva o neúspěchu
    \item \inlinecode{MSG\_OK} -- zpráva o úspěchu nebo potvrzení
    \item \inlinecode{MSG\_RUNTEST} -- pokyn k započnutí testu
    \item \inlinecode{MSG\_STOP} -- pokyn k ukončení testování
    \item \inlinecode{MSG\_EXCEPTION} -- chybná zpráva, slouží pro rozpoznání neplatné zprávy
\end{itemize}

Data zprávy jsou uložena v dynamickém poli. 

\todo{Doplnit}


\section{Testovací služba}
Jak již bylo zmíněno, jádrem k ovládání testování je testovací služba. V implementaci je tento celek rozdělen na dvě hlavní třídy. Třída \inlinecode{TestService} obstarává jednotlivé úkony služby, následně třída \inlinecode{ServiceRunner} propojuje třídu \inlinecode{TestService} s ostatními komponentami, které jsou potřebné pro testování.

\subsection{Nastavení služby}

Konfigurace služby je uložena v souboru \inlinecode{config.xml}. Tento soubor obsahuje tři hodnoty pro nastavení:

\begin{itemize}
    \item \inlinecode{ip} -- Adresa, na které bude služba poslouchat příchozí připojení. Všechna zařízení se budou připojovat na tuto adresu.
    \item \inlinecode{port} -- Síťový port, skrz který služba provádí komunikaci
    \item \inlinecode{participants} -- Počet fyzických participantů, které se připojí ke službě.   
\end{itemize}

Ukázku tohoto souboru můžeme vidět na výpisu \ref{listing:configxml}. V knihovně se o konfiguraci stará třída \inlinecode{Configuration}. Tato třída existuje v jmenném prostoru \inlinecode{GlobalVar}. Tento jmenný prostor simuluje globální proměnnou. Kvůli tomu jsou ale všechny třídy, které jsou odsud používány, konstruovány tak, aby primárně fungovaly jen ke čtení. Při vytvoření instance třída získá konfiguraci z konfiguračního souboru a uloží je do vnitřních proměnných třídy. Tyto proměnné jsou určeny pouze ke čtení, nelze je upravovat. Stav konfigurace určuje proměnná \inlinecode{bool IsValid}. V případě chyby je tato proměnná nastavená na hodnotu \inlinecode{false} a v proměnné \inlinecode{Exception} je uložen důvod neúspěchu. V opačném případě je tato proměnná nastavena na hodnotu \inlinecode{true}.

\begin{listing}[htbp]
    \centering
    \begin{minted}{xml}
        <?xml version="1.0" encoding="UTF-8" ?>
        <configuration>
          <!-- IP of the service-->
          <ip>192.168.4.100</ip>
          <!-- Port of the service -->
          <port>1337</port>
          <!-- Number of non-virtualized participants-->
          <participants>1</participants>
        </configuration>
    \end{minted}
    \caption{Ukázka konfiguračního souboru}
    \label{listing:configxml}
\end{listing}


\subsection{Operace služby}

Jak jsem již zmínil, jednotlivé úkony služby jsou implementovány v třídě \inlinecode{TestService}. Třída po své konstrukci inicializuje vnitřní proměnné, ale neprovádí žádné úkony. 

\subsubsection{Odesílání a přijímání zpráv}
Struktura jedné zprávy je již známým faktem. O jednotlivá připojení se avšak stará třída \inlinecode{TestService}. Třída má dvě hlavní statické metody pro odesílání a přijímání jednotlivých zpráv. Tyto metody jsou:

\begin{itemize}
    \item \inlinecode{static void SendMessage(TcpClient client, Message msg)} \\
    Metoda pro odeslání zprávy jednomu klientovi
    \item \inlinecode{static bool RcvMessage(TcpClient client, out Message msg)} \\
    Metoda pro přijmutí zprávy od jednoho klienta
\end{itemize}

Všechny zprávy jsou následně odesílány a přijímány za pomoci těchto dvou metod. Rozšířením těchto metod jsou metody \inlinecode{SendParticipantsMessage} a \inlinecode{CheckParticipantsResponse}. Obě metody pracují se všemi participanty připojenými ke službě. Metoda \inlinecode{SendParticipantsMessage} odešle všem participantům jednu zprávu, naopak metoda \inlinecode{CheckParticipantsResponse} zprávu od všech participantů přijme a zkontroluje, zda se všechny typy zprávy shodují s očekávaným typem zprávy. Očekávaný typ zprávy je předán v argumentu metody. \todo{CheckParticipantsResponse co vrací}

\subsubsection{Inicializace}

Metodou \inlinecode{bool Init(int InitTimeout)} třída \inlinecode{TestService} inicializuje běh služby. Proměnná \inlinecode{InitTimeout} určuje dobu, kdy služba čeká na příchozí komunikaci. Doba je metodě předávána, stejně jako všem ostatním metodám, které mají definovaný časový limit, v milisekundách. 

Metoda z konfigurace zjistí IP adresu a port, na kterém má služba běžet. Následně začne na této IP adrese a portu poslouchat příchozí komunikaci. V této fázi se připojují pouze fyzická zařízení. Z nastavení služba ví, kolik participantů má očekávat. 

V případě příchozí komunikace je poté zavolána metoda \inlinecode{AddClient}. Tato metoda vytvoří klienta a přijme identifikační zprávu od participanta. Tato zpráva má typ zprávy \inlinecode{MSG\_OK} a v datech zprávy obsahuje MAC adresu participanta. Metoda odešle participantovi potvrzovací zprávu s typem zprávy \inlinecode{MSG\_OK}, bez dat. 

Po úspěšném ověření připojení metoda uloží vytvořeného klienta a jeho MAC adresu do seznamu připojených participantů. Metoda zároveň rozezná připojení virtuálního participanta. Tento participant má MAC adresu \inlinecode{DE:AD:BE:EF:00:00}. Nakonec metoda vrací tři hodnoty typu integer:
\begin{itemize}
    \item \inlinecode{0} -- pokud připojování participanta vyústilo v neúspěch, ať už kvůli překročení časového limitu, nebo kvůli nedodržení stanovené komunikace.
    \item \inlinecode{1} -- pokud služba úspěšně přijme fyzického participanta
    \item \inlinecode{2} -- pokud služba úspěšně přijme virtuálního participanta
\end{itemize}

Služba úspěšně ukončuje inicializační fázi poté co se úspěšně připojí stejný počet participantů, jako bylo určeno v konfiguraci. V případě nepřipojení se očekávaného počtu zařízení v definovaném čase, obdržení špatné nebo žádné zprávy služba vyhodnocuje inicializační fázi jako neúspěšnou. Připojení virtualizovaného participanta v této fázi vyústí též v neúspěch inicializace.

\subsubsection{Spuštění testu}

Nejpodstatnější operací je spuštění jednotlivých testů. O toto se stará metoda \inlinecode{TestResult RunTest<T>(T testEnum, int timeout)}. Metoda očekává dva parametry:

\begin{itemize}
    \item \inlinecode{T testEnum} -- Identifikátor testu s generickým identifikátorem T, který typem musí být enumerátor.
    \item \inlinecode{int timeout} -- Maximální délka, po kterou služba očekává odpověď od participantů testu. 
\end{itemize}

Pro započnutí testu na testovaném zařízení služba odešle zprávu, s typem zprávy \inlinecode{MSG\_RUNTEST}. Data této zprávy obsahují:

\begin{enumerate}
    \item Číselnou reprezentaci enumerátoru \inlinecode{testEnum}. Ten je uložen v prvních 4 bytech dat zprávy.
    \item Číselnou reprezentaci enumerátoru \inlinecode{TestStateE}, určující stádium testování. Uložen je ve dvou bajtech dat zprávy hned za identifikátorem testu.
\end{enumerate}

Tato metoda postupně spouští všechny stádia testování. Tyto stádia jsou rozlišovány za pomocí emulátoru \inlinecode{TestStateE}. Jednotlivě:

\begin{itemize}
    \item \inlinecode{TEST\_STARTUP} -- příprava na testování
    \item \inlinecode{TEST\_RUN} -- spuštění testu
    \item \inlinecode{TEST\_TEARDOW} -- úklid po testu
\end{itemize}

Po spuštění jednotlivých stádií testování metoda čeká na odpověď od všech participantů. Participant odesílá službě zprávu s typem \inlinecode{MSG\_OK} pro úspěch, a nebo s typem \inlinecode{MSG\_FAIL} pro neúspěch. Obě zprávy neobsahují žádné data.

Doba čekání je určena argumentem \inlinecode{timeout}. Tento čas je vázaný na jednotlivá stádia testování. Tedy pokud metoda má časový limit 30 sekund, tak poté bude po každém stádiu testování čekat 30 sekund na odpověď od participanta. V případě že vyprší časový limit, tak je participant považován jako v chybném stavu. Taktéž je participant považován jako v chybném stavu, pokud vrátí neúspěch u přípravy na testování nebo u úklidu po testu.

Metoda následně vrací jednotu z hodnot enumerátoru \inlinecode{TestResult}. Tyto hodnoty jsou:

\begin{itemize}
    \item \inlinecode{TEST\_SUCCESS} -- test proběhl úspěšně
    \item \inlinecode{TEST\_FAIL} -- test proběhl neúspěšně
    \item \inlinecode{TEST\_ERROR} -- v průběhu testování se vyskytla chyba a další testování není možné   
\end{itemize}


\subsubsection{Ukončení testování}

Po dokončení celého testovacího běhu je zavolána metoda \inlinecode{Stop()}. Tato metoda odešle všem participantům stále připojeným ke službě zprávu s typem \inlinecode{MSG\_STOP} a ukončí spojení. Testování avšak může být ukončeno i v případě vzniklé chyby. Služba se snaží v případě chyby ukončit co nejvíce zařízení skrz stanovený protokol.

\subsubsection{Virtuální participanti}

Jelikož každý test může obsahovat různý počet virtuálních participantů, je podstatné, aby služba mohla při testovacím běhu tyto participanty přidávat a odebírat. K tomuto slouží dvě metody. Metoda \inlinecode{AddVirtualParticipant} řekne službě, že má očekávat připojení virtuálního participanta. V případě úspěšného připojení virtuálního participanta služba vrací úspěch. Naopak případě chyby, nebo připojení fyzického participanta, metoda vrací neúspěch. O opačnou operaci se stará metoda \inlinecode{StopVirtualDevices}. Metoda po svém zavolání odešle všem virtuálním participantům zprávu o ukončení testování a tato zařízení jsou odpojena a ukončena.

\subsection{Propojení s ostatními komponentami}

O propojení třídy \inlinecode{TestService} s ostatními komponentami knihovny se stará třída \inlinecode{ServiceRunner}.
Tato třída ve svém konstruktoru inicializuje konfiguraci služby a zjistí, zda je validní. Tento konstruktor přijímá jeden argument -- časový limit na inicializaci. V případě nezadání tohoto parametru třída využije defaultní časový limit definovaný ve třídě na 60 sekund. Následně konstruktor vytvoří instanci třídy \inlinecode{TestService} a zavolá metodu \inlinecode{Init}. V případě jakékoliv chyby je služba vyhodnocena jako v chybovém stavu. Toto určuje enumerátor \inlinecode{RunnerState}, který má dvě možné hodnoty: 

\begin{itemize}
    \item \inlinecode{ERROR} -- služba je v chybovém stavu
    \item \inlinecode{OK}  -- služba je v pořádku  
\end{itemize}

Tato informace je uložena ve vlastní proměnné \inlinecode{State}. Následný důvod vyhodnocení chyby je uloženo v proměnné \inlinecode{Exception}. Třída následně má další metody, skrz které umožňuje ovládání instance třídy \inlinecode{TestService}. Tyto metody jsou:

\begin{itemize}
    \item \inlinecode{bool RunTest<T>(T test, int timeout = DefaultTimeout)} \\ Metoda předá instrukci pro spuštění testu. Očekává stejné parametry jako metoda \inlinecode{RunTest} třídy \inlinecode{TestService}. Jedinou změnou je, že pokud časový limit nebude zadán, tak bude použit defaultní časový limit.
    \item \inlinecode{bool AddVirtualParticipant()} \\ Metoda předá informaci o očekávání připojení virtuálního participanta.
    \item \inlinecode{void TestCleanup()} \\ Metoda je volána po dokončení jednotlivého testu. V aktuální implementaci třída předá informaci o požadavku na odpojení virtuálních participantů.
    \item \inlinecode{void Stop()} \\ Metoda předá informaci o ukončení testovacího běhu.
\end{itemize}


\section{Rozhraní pro testovaná zařízení}

Rozhraní pro testovaná zařízení umožňuje propojení s testovací službou. Jeho cílem je být co nejjednodušší pro co nejrychlejší implementaci. Definici rozhraní pro jazyk \inlinecode{C++} můžeme vidět na výpisu \ref{listing:dev_if}.

\begin{listing}[htbp]
    \begin{minted}[breaklines]{cpp}
    class DeviceIf
    {
    public:

        virtual ~DeviceIf() {}

        virtual bool createConnection(const char * ipAddress, uint16_t port) = 0;

        virtual bool sendMessage(const Message& msg) = 0;

        virtual bool rcvMessage(Message& msg) = 0;

        virtual TestCaseIf * getTest(uint32_t test) = 0;

        virtual bool getMacAddress(uint8_t** macAddr, uint32_t* size) = 0;
    };
    \end{minted}
\caption{Ukázka definice rozhraní}
\label{listing:dev_if}
\end{listing}

\todo{Dopsat}


\section{Virtuální participant}
Implementace virtuálního participanta přímo kopíruje navrhnuté rozhraní pro testované zařízení. Logika virtuálního participanta je obsažena ve třídě \inlinecode{VirtualClient}. Tato třída tři možné konstruktory:

\begin{itemize} 
    \item \inlinecode{VirtualClient(ITestClient device)} \\
    Konstruktor dostane v argumentu odkaz na implementaci rozhraní zařízení
    \item \inlinecode{VirtualClient (ITestCase testCase)} \\
    Konstruktor dostane v argumentu odkaz na implementaci jednoho testu, který je odvozený z rozhraní pro test.
    \item \inlinecode{VirtualClient (Func<UInt32, ITestCase> getTest)} \\
    Argumentem konstruktoru je funkce, která obsahuje seznam testů. Funkce v argumentu dostane číselnou reprezentaci testu a vrací instanci testu, který je odvozený z rozhraní testu.
\end{itemize}

Třída následně vytvoří instanci implementace rozhraní klienta. Tato implementace je obsažena ve třídě \inlinecode{TestClientDefault}. Testovací běh virtuálního participanta funguje na stejném principu jako běh ostatních testovaných zařízení. Rozdílem je, že třída \inlinecode{VirtualClient} vytvoří nové vlákno, na kterém běží tento virtuální participant. Vytvořené vlákno (a s ním i virtuální participant) je poté spuštěno metodou \inlinecode{Start}. Následně metoda \inlinecode{Stop} ukončuje běh virtuálního participanta. Metoda nejdříve čeká, zda se participant a vlákno ukončí standardním způsobem. Pokud se virtuální participant neukončí ve stanoveném časovém limitu, tak ho metoda ukončí ukončením běžícího vlákna.
\todo{Doplnit}


\section{Propojení s frameworkem MSTest}

Podstatnou součástí implementace je propojení služby s testovacím frameworkem MSTest. O toto se stará třída \inlinecode{API}. Jako jediná třída využívá nástroje tohoto frameworku. Třída je koncipovaná jako statická třída, i když je závislá na jím vytvořené instanci třídy \inlinecode{ServiceRunner}. Toto je uděláno z důvodu designového omezení frameworku MSTest. 

Při spuštění jednotlivých testů je nutné brát v potaz, že testovací služba neví jaké testy budou kdy spuštěny. Testovací služba se tedy musí používat nezávisle na těchto testech. K tomuto využijeme dvě funkce frameworku MSTest -- \inlinecode{AssemblyInitialize} a \inlinecode{AssemblyCleanup}. Tyto dvě funkce jsou spuštěny ještě před započnutím a po skončení testovaní. V třídě \inlinecode{API} jejím implementacím odpovídají metody \inlinecode{AssemblyInit} a \inlinecode{AssemblyCleanup}. Metoda \inlinecode{AssemblyInit} přijímá dva argumenty:

\begin{itemize}
    \item \inlinecode{Assembly assembly} -- odkaz na runtime blok, který reprezentuje spuštěný blok programu mimo knihovnu
    \item \inlinecode{int InitTimeout} -- časový limit na inicializační fázi služby (tedy časový limit, který je využíván v metodě \inlinecode{TestService.Init}), pokud je ponechán prázdný, je použit defaultní časový limit z třídy \inlinecode{ServiceRunner}.
\end{itemize}

Předání runtime bloku -- neboli assembly -- je sice dobrovolným krokem, ale podstatným. Při odesílání testů služba odesílá pouze číselnou reprezentaci enumerátoru, který test identifikuje. Tento enumerátor ale nemusí být stejného typu. Proto zde může vzniknout kolize.

Framework MSTest využívá pro rozlišení testovacích součástí tzv. atributy. Atributy přidávají metadata k třídám, typům, metodám atd. \cite{attribute_docs}. Framework MSTest využívá například tyto atributy:

\begin{itemize}
    \item \inlinecode{AssemblyInitialize} -- identifikuje inicializační funkci
    \item \inlinecode{AssemblyCleanup} -- identifikuje funkci která bude spuštěna po skončení všech testů
    \item \inlinecode{TestClass} -- identifikuje třídu, která obsahuje testy
    \item \inlinecode{TestMethod} -- identifikuje metody, které reprezentují jednotlivé testy
\end{itemize}

Testovací knihovna tuto sadu atributů rozšiřuje o atribut \inlinecode{TestEnum}. Tento atribut říká, že enumerátor s atributem \inlinecode{TestEnum} reprezentuje nějaké testy. Proto metoda \inlinecode{AssemblyInit} prostřednictvím argumentu \inlinecode{assembly} zjistí, které enumerátory mají tento atribut. Následně zkontroluje, zdali tyto enumerátory nemají mezi sebou kolize. V případě nalezení kolize je vyhozena výjimka. Tato kontrola se dá přeskočit předáním hodnoty \inlinecode{null} v argumentu \inlinecode{assembly}.

Metoda \inlinecode{AssemblyCleanup} symetricky ukončuje běh testovací služby. V určitých případech tato funkce nemusí být zavolána \question{Je zde potřeba citace}. Při zjištění chyby v běhu, která vyústí v ukončení testovacího běhu, je tato funkce volána manuálně.

Pro spuštění testování je použita metoda \inlinecode{Run}. Tato metoda přijímá stejné argumenty jako metoda \inlinecode{RunTest} třídy \inlinecode{ServiceRunner}. Metoda zkontroluje, že argument identifikující test je typem enumerátor a obsahuje atribut \inlinecode{TestEnum}. V případě že se testovací služba nenachází v chybném stavu, tak metoda předá testovací službě direktivu k započnutí testu. Následně vyhodnotí úspěšnost testu. V případě, že služba je v chybném stavu, metoda vyhodnotí test jako bezvýsledný. 
 

\chapter{Demonstrace použití knihovny}

V této kapitole se budu věnovat demonstraci použití knihovny při testování. K tomu použiji jeden z průmyslových protokolů.

\section{Využití průmyslových protokolů}
Jak jsem již definoval v sekci \ref{sec:fieldbus}, protokoly EtherNet/IP a ModbusTCP mají jasně stanovenou komunikaci. Jejich implementace avšak ale není triviální. Proto k vytvoření propojení na základě těchto protokolů použiji již dostupné open-source knihovny.  

\subsection{Průzkum open-source knihoven}
K tomu abych mohl správně ohodnotit dostupné open-source knihovny je potřeba definovat kritéria, které musí knihovna splňovat. Jednotlivé knihovny musí být kompatibilní s vytvořenou testovací knihovnou. Tedy, knihovny by měli být implementovány v jazyce \csharp{} a kompatibilní s .NET Framework 4.8. 

Vybrané knihovny budu hodnotit na základě několika faktorů. Mezi tyto faktory bude patřit:

\begin{itemize}
    \item Datum poslední aktualizace
    \item Náročnost využití knihovny, tedy náročnost implementace do knihovny a náročnost následného použití knihovny, hodnocené na stupnici 1--10, kde 1 značí nejjednodušší a 10 nejtěžší 
    \item Hodnocení dokumentace na stupnici 0--10, kde 0 značí, že knihovna neobsahuje žádnou dokumentaci a 10 značí nejkvalitnější dokumentaci
\end{itemize}

Následně na základě těchto kritérií a mém osobním názoru doporučím knihovnu vhodnou k použití s testovací knihovnou. 

\subsubsection{ModbusTCP}

V tabulce \ref{tab:modbus} můžeme vidět seznam nejvhodnějších knihoven pro využití v knihovně. Dle jednotlivých hodnocení můžeme vidět že knihovny se pohybují okolo stejné náročnosti na využití v knihovně. Všechny knihovny jsou dostupné jako NuGet balíčky, což jejich implementaci velice usnadňuje.

Nejpoužívanější knihovnou, dle počtu stažení NuGet balíčku, je knihovna NModbus. Knihovna má průměrnou dokumentaci, kde popisuje jednotlivé třídy. Tato knihovna je udržována komunitně, a již dvakrát byla opuštěna předchozími autory. V kontextu toho, že knihovna má nejstarší datum poslední aktualizace, zde vzniká otázka, zda knihovna bude v budoucnu udržována.

Stejně jako knihovna NModbus, knihovny FluentModbus a Modbus jsou komunitním projektem. Knihovna Modbus ale bohužel neobsahuje žádnou dokumentaci až na krátkou ukázku kódu. Proto má knihovna v hodnocení zvýšenou náročnost na využití kvůli chybějící dokumentaci. Oproti tomu knihovna FluentModbus obsahuje nejlepší dokumentaci ze všech zmíněných knihoven. Bohužel knihovna FluentModbus aktuálně nepodporuje všechny funkce protokolu ModbusTCP. 

Jako nejvhodnější knihovnu jsem tedy vyhodnotil knihovnu EasyModbusTCP. Tato knihovna jako jediná je vytvořena firmou Rossmann Engineering. Z tohoto důvodu předpokládám, že na knihovnu jsou vyvíjeny vyšší kvalitativní nároky, než na ostatní knihovny.  

\begin{table}[H]\centering
    \resizebox{\textwidth}{!}{
        \begin{tabular}{|C{3cm}|C{3cm}|C{2.1cm}|C{1.8cm}|C{2.3cm}|C{4cm}|}\hline
            Název & Autor & Poslední aktualizace & Náročnost využití & Hodnocení dokumentace &  Dostupné na adrese \\\hline 
            EasyModbusTCP & Rossmann Engineering & 31.12.2020 & 1 & 5 & \url{easymodbustcp.net}\\\hline
            FluentModbus & Apollo3zehn, et al. & 13.04.2021 & 1 & 6 & \url{github.com/Apollo3zehn/FluentModbus} \\\hline 
            NModbus & Rich Quackenbush, et al. & 14.07.2020 & 2 & 4 & \url{github.com/NModbus/NModbus} \\\hline 
            Modbus & Andres Müller, et al. & 13.04.2021 & 3 & 1 & \url{github.com/AndreasAmMueller/Modbus} \\\hline
        \end{tabular}
    }
    \caption{Seznam dostupných knihoven pro protokol ModbusTCP}
    \label{tab:modbus}
\end{table}

\subsubsection{EtherNet/IP}



\subsection{Použití knihovny} 

\chapter{Vyhodnocení vytvořeného řešení}\label{chap:evaluation}

\dummytext{1-10}

\begin{conclusion}
\textcolor{gray}{\lipsum[1-2]}
\end{conclusion}


{
    \setlength{\emergencystretch}{3em} 
    \printbibliography
}

\appendix

\chapter{Seznam použitých zkratek}
\begin{description}
	\item[XML] Extensible markup language
\end{description}

\chapter{Seznam cen zařízení a zdrojů}

V tabulce \ref{tab:device_prices} lze vidět seznam zařízení, jejich cenu a zdroj ceny. Ceny byli získány 16.04.2021 a případně převedeny do českých korun dle kurzu k tomuto dni.

\begin{table}[H]
    \centering
    \resizebox{\textwidth}{!}{
    \begin{tabular}{|C{4cm}|C{2.5cm}|C{9.5cm}|}\hline
        \textbf{Produkt} & \textbf{Cena za kus bez DPH} & \textbf{Zdroj} \\\hline
        Beckhoff CX2030-0125 &  \koruna{66650.44} &  \url{https://www.radwell.co.uk/Buy/BECKHOFF/BECKHOFF/CX2030-0125}\\\hline
        Beckhoff CX2100-0004 &  \koruna{8663.21} &  \url{https://www.radwell.co.uk/en-GB/Buy/BECKHOFF/BECKHOFF/CX2100-0004/} \\\hline
        Schneider Electric BMXP342020 340-20  & \koruna{30536.8} & \url{https://cz.farnell.com/schneider-electric/bmxp342020/processor-module-0-095a-24vdc/dp/2835381} \\\hline
        Schneider Electric BMXCPS2000 &  \koruna{5220.27} &  \url{https://cz.rs-online.com/web/p/prislusenstvi-pro-plc/0148026/} \\\hline
        Schneider Electric BMXXBP0400 &  \koruna{3493.03} &  \url{https://uk.rs-online.com/web/p/plc-accessories/0147922/}  \\\hline
        Schneider Electric M251MESC &  \koruna{8192.46} &   \url{https://cz.rs-online.com/web/p/plc-programovatelne-logicke-kontrolery/8066755/} \\\hline
        Schneider Electric TM262L20MESE8T &  \koruna{36636.86}  &  \url{https://at.rs-online.com/web/p/sps-zentralbaugruppen/2011448/} \\\hline
        Rockwell Automation 1769-L24ER-QB1B &  \koruna{38526.54} & \url{https://www.routeco.com/en-gb/shop/programmable-controllers/compactlogix/1769-l24er-qb1b} \\\hline
        Rockwell Automation 1783-BMS20CGN & \koruna{64480.27} &  \url{https://cz.wiautomation.com/allen-bradley/industrial-communication/stratix/1783BMS20CGN} \\ \hline
        Schneider Electric Machine Expert & \koruna{23987.18} & \url{https://cz.wiautomation.com/schneider-electric/software/ESECAPCZZSPMZZ} \\\hline
        SO Machine & \koruna{26907.28} & \url{https://www.alliedelec.com/product/schneider-electric/somnacczxspazz/70596} \\\hline
        Studio 5000 & \koruna{79140.18} &  \url{https://twcontrols.com/lessons/where-can-i-download-studio-5000-logix-designer-rslogix-5000-and-what-is-the-pricedifference-of-each-version}\\\hline
        Unity Pro & \koruna{102299.92} &  \url{https://cz.wiautomation.com/schneider-electric/software/UNYUPDSPUXUG} \\\hline
    \end{tabular}
    }
    \caption{Seznam zdrojů cen zařízení a softwarových licencí}
    \label{tab:device_prices}
\end{table}

\input{assets/appendix/medium.tex}

\end{document}
